\documentclass[a4paper,11pt] {article}
\usepackage{amsmath, amssymb, amsthm}
\usepackage[pdftex] {graphicx}
\usepackage{verbatim}
\usepackage{fullpage, setspace}
\usepackage{stmaryrd}
\usepackage{mathtools}
\usepackage{extarrows}
\usepackage{supertabular}
\usepackage{units}

\theoremstyle{plain}
\newtheorem{theorem}{Theorem}[section]
\newtheorem*{theorem*}{Theorem}
\newtheorem{definition}[theorem]{Definition}

\newcommand{\negjoinrel}{\mathrel{\mkern3mu}}
\newcommand{\rlaprel}[1]{\mathrel{\mathrlap{#1}}}
\newcommand{\rel}{\rlaprel{\leftarrow}\negjoinrel\rightarrow}
\newcommand{\trel}{\rlaprel{\leftarrow}\negjoinrel\rlaprel{\leftarrow}\rightarrow}
\newcommand{\srel}{\rlaprel{\leftarrow}\rlaprel{\rightarrow}\negjoinrel\rightarrow}
\newcommand{\tsrel}{\rlaprel{\leftarrow}\negjoinrel\rlaprel{\leftarrow}
  \joinrel\rlaprel{\rightarrow}\negjoinrel\rightarrow}

\newcommand{\lget}[1]{\textit{$#1$.get}}
\newcommand{\lput}[1]{\textit{$#1$.put}}
\newcommand{\lcreate}[1]{\textit{$#1$.create}}

\newcommand{\lcopy}{\textit{copy}}
\newcommand{\lconst}{\textit{const}}
\newcommand{\lins}{\textit{ins}}
\newcommand{\ldel}{\textit{del}}

\newcommand{\lpermute}{\textit{permute}}

\newcommand{\lqdel}{\textit{qdel}}
\newcommand{\lqins}{\textit{qins}}
\newcommand{\lqconst}{\textit{qconst}}
\newcommand{\lqdup}{\textit{dup}}

\newcommand{\lquot}{\textit{lquot}}
\newcommand{\rquot}{\textit{rquot}}

\newcommand{\lcanonize}[1]{\textit{$#1$.canonize}}
\newcommand{\lchoose}[1]{\textit{$#1$.choose}}

\newcommand{\lcanonizelens}{\textit{canonizer\_of\;}}
\newcommand{\lnormalize}{\textit{normalize}}

\newcommand{\lcolumnize}{\textit{columnize}}
\newcommand{\lsort}{\textit{sort}}


\newcommand{\canon}{\rlaprel{\longleftarrow}\rlaprel{\longrightarrow}
  \negjoinrel\rightarrow\;\,}

%\newcommand{\lensbetween}{\xLongleftrightarrow{}}
\newcommand{\lensbetween}[1]{\xLongleftrightarrow{#1}}
%\newcommand{\lensbetween}{\iff}
\newcommand{\perm}{ \textbf{perm}\; }
\newcommand{\with}{ \;\textbf{with}\; }

\newcommand{\defn}[1]{\textbf{#1}}
\newcommand{\name}[1]{\textsf{#1}}

\newcommand{\niceFrac}[2]{%
    \raise.5ex\hbox{$#1$}%
    \kern-.1em/\kern-.15em%
    \lower.25ex\hbox{$#2$}}

\begin{document}
\title{Lens Synthesis}
\date{}
\maketitle

% generated by Ott 0.25 from: qre.ott
\newcommand{\ottdrule}[4][]{{\displaystyle\frac{\begin{array}{l}#2\end{array}}{#3}\quad\ottdrulename{#4}}}
\newcommand{\ottusedrule}[1]{\[#1\]}
\newcommand{\ottpremise}[1]{ #1 \\}
\newenvironment{ottdefnblock}[3][]{ \framebox{\mbox{#2}} \quad #3 \\[0pt]}{}
\newenvironment{ottfundefnblock}[3][]{ \framebox{\mbox{#2}} \quad #3 \\[0pt]\begin{displaymath}\begin{array}{l}}{\end{array}\end{displaymath}}
\newcommand{\ottfunclause}[2]{ #1 \equiv #2 \\}
\newcommand{\ottnt}[1]{\mathit{#1}}
\newcommand{\ottmv}[1]{\mathit{#1}}
\newcommand{\ottkw}[1]{\mathbf{#1}}
\newcommand{\ottsym}[1]{#1}
\newcommand{\ottcom}[1]{\text{#1}}
\newcommand{\ottdrulename}[1]{\textsc{#1}}
\newcommand{\ottcomplu}[5]{\overline{#1}^{\,#2\in #3 #4 #5}}
\newcommand{\ottcompu}[3]{\overline{#1}^{\,#2<#3}}
\newcommand{\ottcomp}[2]{\overline{#1}^{\,#2}}
\newcommand{\ottgrammartabular}[1]{\begin{supertabular}{llcllllll}#1\end{supertabular}}
\newcommand{\ottmetavartabular}[1]{\begin{supertabular}{ll}#1\end{supertabular}}
\newcommand{\ottrulehead}[3]{$#1$ & & $#2$ & & & \multicolumn{2}{l}{#3}}
\newcommand{\ottprodline}[6]{& & $#1$ & $#2$ & $#3 #4$ & $#5$ & $#6$}
\newcommand{\ottfirstprodline}[6]{\ottprodline{#1}{#2}{#3}{#4}{#5}{#6}}
\newcommand{\ottlongprodline}[2]{& & $#1$ & \multicolumn{4}{l}{$#2$}}
\newcommand{\ottfirstlongprodline}[2]{\ottlongprodline{#1}{#2}}
\newcommand{\ottbindspecprodline}[6]{\ottprodline{#1}{#2}{#3}{#4}{#5}{#6}}
\newcommand{\ottprodnewline}{\\}
\newcommand{\ottinterrule}{\\[5.0mm]}
\newcommand{\ottafterlastrule}{\\}
\newcommand{\ottmetavars}{
\ottmetavartabular{
 $ \ottmv{ascii\_string} ,\, \ottmv{s} $ &  \\
 $ \ottmv{index} ,\, \ottmv{n} $ &  \\
}}

\newcommand{\ottr}{
\ottrulehead{\ottnt{r}}{::=}{}\ottprodnewline
\ottfirstprodline{|}{\epsilon}{}{}{}{\ottcom{empty string}}\ottprodnewline
\ottprodline{|}{\ottmv{s}}{}{}{}{\ottcom{ASCII string}}\ottprodnewline
\ottprodline{|}{\ottsym{r*}}{}{}{}{\ottcom{Kleene Star}}\ottprodnewline
\ottprodline{|}{\ottnt{r_{{\mathrm{1}}}} \, \cdot \, \ottnt{r_{{\mathrm{2}}}}}{}{}{}{\ottcom{Concatenation}}\ottprodnewline
\ottprodline{|}{\ottnt{r_{{\mathrm{1}}}} \, | \, \ottnt{r_{{\mathrm{2}}}}}{}{}{}{\ottcom{Union}}}

\newcommand{\ottq}{
\ottrulehead{\ottnt{q}}{::=}{}\ottprodnewline
\ottfirstprodline{|}{\ottnt{r}}{}{}{}{\ottcom{RE}}\ottprodnewline
\ottprodline{|}{\ottnt{r}  \mapsto  \ottmv{s}}{}{}{}{\ottcom{Quotient}}\ottprodnewline
\ottprodline{|}{\ottkw{perm} \, \ottsym{(}  \ottnt{q_{{\mathrm{1}}}}  \ottsym{...}  \ottnt{q_{\ottmv{n}}}  \ottsym{)} \, \ottkw{with} \, \ottnt{q}}{}{}{}{\ottcom{Permutation}}\ottprodnewline
\ottprodline{|}{\ottsym{q*}}{}{}{}{\ottcom{Kleene Star}}\ottprodnewline
\ottprodline{|}{\ottnt{q_{{\mathrm{1}}}} \, \cdot \, \ottnt{q_{{\mathrm{2}}}}}{}{}{}{\ottcom{Concatenation}}\ottprodnewline
\ottprodline{|}{\ottnt{q_{{\mathrm{1}}}} \, | \, \ottnt{q_{{\mathrm{2}}}}}{}{}{}{\ottcom{Union}}}

\newcommand{\ottterminals}{
\ottrulehead{\ottnt{terminals}}{::=}{}\ottprodnewline
\ottfirstprodline{|}{ \epsilon }{}{}{}{}\ottprodnewline
\ottprodline{|}{ \mapsto }{}{}{}{}\ottprodnewline
\ottprodline{|}{ \cdot }{}{}{}{}\ottprodnewline
\ottprodline{|}{ | }{}{}{}{}}

\newcommand{\ottformula}{
\ottrulehead{\ottnt{formula}}{::=}{}\ottprodnewline
\ottfirstprodline{|}{\ottnt{judgement}}{}{}{}{}}

\newcommand{\ottjudgement}{
\ottrulehead{\ottnt{judgement}}{::=}{}}

\newcommand{\ottuserXXsyntax}{
\ottrulehead{\ottnt{user\_syntax}}{::=}{}\ottprodnewline
\ottfirstprodline{|}{\ottmv{ascii\_string}}{}{}{}{}\ottprodnewline
\ottprodline{|}{\ottmv{index}}{}{}{}{}\ottprodnewline
\ottprodline{|}{\ottnt{r}}{}{}{}{}\ottprodnewline
\ottprodline{|}{\ottnt{q}}{}{}{}{}\ottprodnewline
\ottprodline{|}{\ottnt{terminals}}{}{}{}{}}

\newcommand{\ottgrammar}{\ottgrammartabular{
\ottr\ottinterrule
\ottq\ottinterrule
\ottterminals\ottinterrule
\ottformula\ottinterrule
\ottjudgement\ottinterrule
\ottuserXXsyntax\ottafterlastrule
}}

% defnss
\newcommand{\ottdefnss}{
}

\newcommand{\ottall}{\ottmetavars\\[0pt]
\ottgrammar\\[5.0mm]
\ottdefnss}



\section{Definitions}
\label{sec:definitions}
Every regular expression $r$ denotes a regular language $\llbracket r \rrbracket$.
A quotient regular expression (QRE) $q$ denotes a equivalence relation
on its whole language $W(q)$ with a set of canonical representives
$K(q) \subseteq W(q)$ called the kernel.
Both $W(q)$ and $K(q)$ are regular
languages, which we represent by regular expressions.

The equivalence relation and canonical representives 
can be described by a canonizer $c_q$
consisting of two functions
\begin{itemize}
\item $\lcanonize{c}: W(q) \to K(q)$ and
\item $\lchoose{c}: K(q) \to W(q)$
\end{itemize}
such that for every $s \in K(q)$,
\begin{align*}
  \lcanonize{c} \; (\lchoose{c} \; s) = s
\end{align*}
We write $c_q: W(q) \canon K(q)$.
As shall be seen, the syntax of QREs makes this associated canonizer obvious.

Define $\sim_q$ on $W(q)$ by $w_1 \sim_q w_2$ iff
$\lcanonize{c_q} \; w_1 = \lcanonize{c_q} \; w_2$.
Then the canonizer $c_q$ for $q$ extends to a quotient lens
$l_q: W(q)/\sim_q \lensbetween{} K(q)$ by having:
\begin{gather*}
  \lget{l_q} \; w = \lcanonize{c_q} \; w\\
  \lput{l_q} \; k \; w = \lchoose{c_q} \; k 
  %\lcreate{l_q} \; k = \lchoose{c_q} \; k
\end{gather*}
The functions satisfy the following lens laws:
\begin{itemize}
\item (\name{GetPut}): $\lput{l_q} \; (\lget{l_q} \; w) \; w \sim_q w$ 
\item (\name{PutGet}): $\lget{l_q} \; (\lput{l_q} \; k \; w) = k$
%\item (\name{CreateGet}): $\lget{l_q} \; (\lcreate{l_q} \; k) = k$
\item (\name{GetEquiv}): If $w \sim_q w'$,
  $\lget{l_q} \; w = \lget{l_q} \; w'$.
\item (\name{PutEquiv}): If $w \sim_q w'$,
  $\lput{l_q} \; k \; w \sim_q \lput{l_q} \; k \; w'$.
  %\item (\name{CreateEquiv}) If $k \sim_q k'$, $\lcreate{l_q} \; k
  %\sim_q \lcreate{l_q} \; k'$.
\end{itemize}
We say that $l_q$ is the lifting of $c_q$.

Since $c_q$ and $l_q$ determine each other in this sense, the denotation of
QREs can be given either way. But the final outcome of synthesis is a
composite lens whose edge lenses satisfy the above lens laws.

\begin{comment}
The equivalence relation can be given by a canonizer $c_q$ consisting of two functions
\begin{itemize}
\item $\lcanonize{c}: W(q) \to K(q)$ and
\item $\lchoose{c}: K(q) \to W(q)$
\end{itemize}
such that for every $s \in K(q)$,
\begin{align*}
  \lcanonize{c} \; (\lchoose{c} \; s) = s
\end{align*}
\end{comment}

\begin{definition}
A quotient regular expression (QRE) has its syntax defined by
the following grammar. Below, $r$ ranges over the usual regular expressions,
$q$ ranges over QREs, and $n \geq 1$.
\begin{center}
\ottgrammartabular{\ottr\ottinterrule}
\end{center}

\begin{center}
\ottgrammartabular{\ottq\ottinterrule}
\end{center}
\end{definition}


A QRE $r$ denotes the identity canonizer; $r \mapsto s$ denotes the
canonizer which maps all strings matching $r$ to the ASCII string $s$,
and just the identity map backwards; $\perm(q_1, \ldots, q_n) \with q$
canonizes by re-ordering the substrings matching a permutation of
$q_1$ through $q_n$ to form the string matching
$q_1 \cdot \ldots \cdot q_n$, and is just the identity backwards;
the others denote canonizers which are homomorphically extended.
To define this formally, we first define $W(q)$ and $K(q)$ between
which the canonizers map, and then state their well-formedness and
validity conditions. 

\begin{definition}
  For $q \in \mathbb{Q}$, the whole language $W(q) \subseteq \Sigma^*$
  is defined as follows:
  \begin{itemize}
  \item $W(r) = \llbracket r \rrbracket$
  \item $W(r \mapsto s) = \llbracket r \rrbracket$
  \item $W(\perm(q_1, \ldots, q_n) \with q) =
    \bigcup_{\sigma \in S_n} \left( \left(\bigodot_{i=1}^{n-1} (W(q_{\sigma(i)})
    \cdot W(q) \right)
    \cdot W(q_{\sigma(n)}) \right)$
  \item $W(q^*) = {W(q)}^*$
  \item $W(q_1 \cdot q_2) = W(q_1) \cdot W(q_2)$
  \item $W(q_1 ~|~ q_2) = W(q_1) ~|~ W(q_2)$
  \end{itemize}
  where Kleene star, concatenation and union for languages are defined
  in the usual way.
  
  Its kernel $K(q) \subseteq W(q) \subseteq \Sigma^*$ is defined as follows:
  \begin{itemize}
  \item $K(r) = r$
  \item $K(r \mapsto s) = \llbracket s \rrbracket$
  \item $K(\perm (q_1, \ldots, q_n) \with q) =
    K(q_1) \cdot K(q) \cdot \ldots \cdot
    K(q) \cdot K(q_n)$  
  \item $K(q^*) = {K(q)}^*$
  \item $K(q_1 \cdot q_2) = K(q_1) \cdot K(q_2)$
  \item $K(q_1 ~|~ q_2) = K(q_1) ~|~ K(q_2)$
  \end{itemize}
\end{definition}

The lenses that QREs denote are well-defined only when their whole
language and kernel meet some semantic conditions.
\begin{definition}
The well-formed judgement on QREs is as follows.
\begin{itemize}
\item Every $r$ is well-formed.
\item $r \mapsto s$ is well-formed if $s \in \llbracket r \rrbracket$.
\item $\perm(q_1, \ldots, q_n) \with q$ is well-formed if every $q_i$ is
  well-formed and:
  \begin{itemize}
  \item For every 
    $s \in W(\perm(q_1, \ldots, q_n) \with q)$, there exists a unique
    $\sigma \in S_n$, unique $s_i \in W(q_i)$ for $1 \leq i \leq n$,
    and unique $t_j \in W(q)$ for $1 \leq j < n$ such that 
    $s = s_{\sigma(1)} \cdot t_1 \cdot \ldots \cdot t_{n-1} \cdot
    s_{\sigma(n)}$.
    
  \item For every $s' \in K(\perm(q_1, \ldots, q_n) \with q)$, there exists
    unique $s_i' \in K(q_i)$, $1 \leq i \leq n$, and unique
    $t_j' \in K(q)$, $1 \leq j < n$, 
    such that $s' = s_1' \cdot t_1' \cdot \ldots \cdot t_{n-1}' \cdot s_n'$.
  \end{itemize}
  
  
\item $q^*$ is well-formed if $q$ is well-formed and:
  \begin{itemize}
  \item For every $s \in W(q^*)$, there exists a unique $n \geq 1$ and
    $s_1, \ldots, s_n \in W(q)$ such that $s = s_1 \cdot \ldots \cdot
    s_n$.
  \item For every $s' \in K(q^*)$, there exists a unique $n \geq 1$ and
    $s_1', \ldots, s_n' \in K(q^*)$ such that
    $s' = s_1' \cdot \ldots \cdot s_n'$.
  \end{itemize}
\item $q_1 \cdot q_2$ is well-formed if $q_1$ and $q_2$ are both well-formed,
  and:
  \begin{itemize}
  \item For every $s \in W(q_1 \cdot q_2)$ there exists unique $s_1 \in W(q_1)$
    and unique $s_2 \in W(q_2)$ such that 
    $s = s_1 \cdot s_2$.
  \item For every $s' \in K(q_1 \cdot q_2)$, there exists unique $s_1' \in K(q_1)$
    and unique $s_2' \in K(q_2)$ such that
    $s' = s_1' \cdot s_2'$.
  \end{itemize}
  
\item $q_1 ~|~ q_2$ is well-formed if both $q_1$ and $q_2$ are well-formed,
  $W(q_1) \cap W(q_2) = \emptyset$, and $K(q_1) \cap K(q_2) = \emptyset$.
\end{itemize}
\end{definition}

The definition of well-formed QREs is exactly such that the following
denotation function is well-defined.
\begin{definition}
  For a well-formed QRE $q$, its denotation
  $\llbracket q \rrbracket = c_q \in W(q) \canon K(q)$ is defined as follows:
  \begin{itemize}
  \item $\llbracket r \rrbracket = \mathit{id}_{\llbracket r \rrbracket}$.
  \item For $q = r \mapsto s$, 
    \begin{align*}
      \lcanonize{c_q}(w) &= s \text{ for all
        $w \in \llbracket r \rrbracket$} \\
      \lchoose{c_q}(s) &= s
    \end{align*}
  \item For $q = \perm(q_1, \ldots, q_n) \with q'$, 
    \begin{align*}
      \lcanonize{c_q}(w_{\sigma(1)} \cdot t_1 \cdot \ldots \cdot t_{n-1}
      \cdot w_{\sigma(n)}) &= w_1 \cdot t_1 \cdot \ldots \cdot t_{n-1} \cdot w_n \\
      \lchoose{c_q}(k_1 \cdot t_1 \ldots \cdot t_{n-1} \cdot k_n)
      &= k_1 \cdot t_1 \cdot \ldots \cdot t_{n-1} \cdot k_n
    \end{align*}
  \item $\llbracket q^* \rrbracket = \llbracket q \rrbracket^* = c$ where 
    \begin{align*}
      \lcanonize{c}(w_1 \cdot \ldots \cdot w_n) &=
      \lcanonize{c_q}(w_1) \cdot \ldots \cdot
      \lcanonize{c_q}(w_n) \\
      \lchoose{c}(k_1 \cdot \ldots \cdot k_n) &=
      \lchoose{c_q}(k_1) \cdot \ldots \cdot
      \lchoose{c_q}(k_n)
    \end{align*}
    for all $n \geq 1$.
  \item For $q = q_1 \cdot q_2$, $\llbracket q \rrbracket = c$ where 
    \begin{align*}
      \lcanonize{c}(w_1 \cdot w_2) &=
      \lcanonize{c_{q_1}}(w_1) \cdot
      \lcanonize{c_{q_2}}(w_2) \\
      \lchoose{c}(k_1 \cdot k_2) &=
      \lchoose{c_{q_1}}(k_1) \cdot
      \lchoose{c_{q_2}}(k_2)
    \end{align*}
  \item For $q = q_1 ~|~ q_2$, $\llbracket q \rrbracket = c$ where
    for all $w_1 \in W(q_1)$, $k_1 \in K(q_1)$, $w_2 \in W(q_2)$,
    and $k_2 \in K(q_2)$,
    \begin{align*}
      \lcanonize{c}(w_1) &=
      \lcanonize{c_{q_1}}(w_1) \\
      \lcanonize{c}(w_2) &=
      \lcanonize{c_{q_2}}(w_2) \\
      \
      \lchoose{c}(k_1) &=
      \lchoose{c_{q_1}}(k_1) \\      
      \lchoose{c}(k_2) &=
      \lchoose{c_{q_2}}(k_2)
    \end{align*}
  \end{itemize}  
\end{definition}
We seek sufficient and efficiently decidable conditions for
well-formedness. The cases for $r$, $r \mapsto s$ and union are trivial;
for Kleene star and concatenation, we can use unambiguous iteration
and unambiguous concatenation respectively. 

\begin{theorem*}
Let 
\end{theorem*}


\begin{theorem*}
  For $\perm(q_1,\ldots, q_n) \with q$, the following are true:
  \begin{itemize}
  \item TODO
  \end{itemize}
\end{theorem*}
\begin{proof}

  

\end{proof}

The following validity judgement
is sufficient for deciding well-formedness.
\begin{definition}
  The validity judgement is defined as follows:
  \begin{itemize}
  \item Every regular expression $r$ is valid.
  \item $r \mapsto s$ is valid if $s$ matches $r$.
  \item $\perm(q_1, \ldots, q_n) \with q$ is valid if all $q_i$ are
    valid and:
    \begin{itemize}
    \item For all $\sigma_1, \sigma_2 \in S_n$,
      \begin{align*}
        (W(q_{\sigma_1(1)}) \cdot W(q) \cdot \ldots
        \cdot W(q) \cdot W(q_{\sigma_1(n)})) \cap
        (W(q_{\sigma_2(1)}) \cdot W(q) \cdot \ldots
        \cdot W(q) \cdot W(q_{\sigma_2(n)})) = \emptyset
      \end{align*}
    \item For all $\sigma \in S_n$,
      $W(q_{\sigma(1)}) \cdot^! W(q) \cdot^!
      \ldots \cdot^! W(q) \cdot^! W(q_{\sigma(n)})$
      is unambiguously concatenable.
    \item $K(q_1) \cdot^! K(q) \cdot^! \ldots
    \cdot^! K(q) \cdot^! K(q_n)$ is unambiguously
    concatenable.
    \end{itemize}
  \item $q^*$ is valid if $q$ is valid and both
    $\bigodot_{i=1}^{!n} W(q)$ and $\bigodot_{i=1}^{!n} K(q)$ are
    unambiguously concatenable for all $n > 1$.
  \item $q_1 \cdot q_2$ is valid if both $q_1$ and $q_2$ are valid,
    $W(q_1) \cdot^{!} W(q_2)$ and $K(q_1) \cdot^{!} K(q_2)$ are unambiguously
    concatenable.
  \item $q_1 ~|~ q_2$ is valid if both $q_1$ and $q_2$ are valid,
    $W(q_1) \cap W(q_2) = \emptyset$, and $K(q_1) \cap K(q_2) = \emptyset$.
  \end{itemize}

\end{definition}
The language can be extended with a custom canonizer by specifying its
denotation, well-formedness condition, and validity condition. Its
validity condition must imply its well-formedness condition.

\begin{theorem}
  Every valid QRE is a well-formed QRE.
\end{theorem}
\begin{proof}
  By induction on QREs.
\end{proof}
Hence every valid QRE $q$ denotes a canonizer $W(q) \canon K(q)$.
Henceforth, all QREs are assumed to be well-formed.
\begin{definition}
  A synthesis goal is 
  \begin{align*}
    q_1 \lensbetween{?} q_2
  \end{align*}
  with examples
  $E = \{(s_1^i, s_2^i)\}_{i=1}^N$, $N$ possibly $0$,
  such that each $s_1^i \in W(q_1)$ and each $s_2^i \in W(q_2)$.
  
  A solution is a quotient lens
  $l: \niceFrac{W(q_1)}{\sim_{q_1}} \lensbetween{} \niceFrac{W(q_2)}{\sim_{q_2}}$
  such that $l = l_{q_1} ; l' ; l_{q_2}^{-1}$, i.e.
  \begin{align*}
    \niceFrac{W(q_1)}{\sim_{q_1}} \lensbetween{l} \niceFrac{W(q_2)}{\sim_{q_2}}
    = \niceFrac{W(q_1)}{\sim_{q_1}} \lensbetween{l_{q_1}} K(q_1) \lensbetween{l'} K(q_2)
    \lensbetween{l_{q_2}^{-1}} \niceFrac{W(q_2)}{\sim_{q_2}}
  \end{align*}
  for some bijective lens
  $l': K(q_1) \lensbetween{} K(q_2)$, and 
  $l$ is consistent with the examples.
\end{definition}
The lens synthesis algorithm just needs to synthesize $l'$
because of the types. The solution $l$ satisfies the following laws:
\begin{itemize}
\item (\name{GetPut}): $\lput{l} \; (\lget{l} \; s_1) \; s_1 \sim_{q_1} s_1$ 
\item (\name{PutGet}): $\lget{l} \; (\lput{l} \; s_2 \; s_1) \sim_{q_2} s_2$
%\item (\name{CreateGet}): $\lget{l} \; (\lcreate{l} \; s_2) \sim_{q_2} s_2$
\item (\name{GetEquiv}): If $s_1 \sim_{q_1} s_1'$, $\lget{l} \; s_1
  \sim_{q_2} \lget{l} \; s_1'$.
\item (\name{PutEquiv}): If $s_2 \sim_{q_2} s_2'$ and $s_1 \sim_{q_1} s_1'$,
  $\lput{l} \; s_2 \; s_1 \sim_{q_1} \lput{l} \; s_2' \; s_1'$.
%\item (\name{CreateEquiv}) If $s_2 \sim_{q_2} s_2'$, $\lcreate{l} \; s_2
%  \sim_{q_2} \lcreate{l} \; s_2'$.
\end{itemize}
For brevity, we omit $\sim_{q}$ from $W(q)/\sim_{q}$ and just write
$l: W(q_1) \lensbetween{} W(q_2)$ when this is clear.

Synthesis goals of the above form give lenses which only canonize at
the edges, so a question is whether ``all'' quotient lenses can be
turned into this form.


\section{Completeness of Synthesis}
All QREs are taken to be well-formed.
\begin{definition}
  A quotient lens
  $\niceFrac{A}{\sim_A} \lensbetween{l} \niceFrac{B}{\sim_B}$
  is synthesizable if there exists QREs $q_1$ and $q_2$ such that
  it is a solution to $q_1 \lensbetween{?} q_2$, i.e.
  \begin{align*}
    \niceFrac{A}{\sim_A} \lensbetween{l} \niceFrac{B}{\sim_B} =
    \niceFrac{W(q_1)}{\sim_{q_1}} \lensbetween{l_{q_1}} K(q_1)
    \lensbetween{l'}
    K(q_2) \lensbetween{l_{q_2}} \niceFrac{W(q_2)}{\sim_{q_2}}
  \end{align*}
  In particular, $A = W(q_1)$, $\sim_A {} = {} \sim_{q_1}$,
  $B = W(q_2)$ and $\sim_B {} = {} \sim_{q_2}$.
\end{definition}
For the following, we consider the semantic space of quotient
lenses in the setting of QREs.
Since QREs denote canonizers,
we have the usual $\lquot$ and $\rquot$ operations:
\begin{itemize}
\item Let $l: \niceFrac{A}{\sim_A} \lensbetween{}
  \niceFrac{B}{\sim_B}$ be a quotient lens and
  $q$ be a QRE such that $K(q) = A$.
  Then
  $l' = \lquot \; q \; l: \niceFrac{W(q)}{\sim} \lensbetween{}
  \niceFrac{B}{\sim_B}$
  is defined by
  \begin{align*}
    \lget{l'} \; w &= \lget{l} \; (\lcanonize{c_q} \; w) \\
    \lput{l'} \; b \; w &=
    \lchoose{c_q} \; (\lput{l} \; b \; (\lcanonize{c_q} \; w)) \; w
  \end{align*}
  and $w \sim w'$ iff $\lcanonize{c_q}(w) \sim_A \lcanonize{c_q}(w')$.
  
\item Let $l: \niceFrac{A}{\sim_A} \lensbetween{}
  \niceFrac{B}{\sim_B}$
  be a quotient lens and
  $q$ be a QRE such that $K(q) = B$.
  Then $l' = \rquot \; l \; q: \niceFrac{A}{\sim_A} \lensbetween{}
  \niceFrac{W(q)}{\sim}$
  is defined by
  \begin{align*}
    \lget{l'} \; a &= \lchoose{c_q} \; (\lget{l} \; a) \\
    \lput{l'} \; w \; a &= \lput{l} \; (\lcanonize{c_q} \; w) \; a
  \end{align*}
  and $w \sim w'$ iff $\lcanonize{c_q} \; w = \lcanonize{c_q} \; w'$.
\end{itemize}
Following the theory of quotient lenses, the operations produce
well-defined quotient lenses; key to this is the fact that the
quotient relation on $K(q)$ for QREs $q$ is always the equality
relation, and this relation is the finest.

Note that as defined, for a left quotient by $q$, the quotient
relation $\sim$ on $W(q)$ is not necessarily the same as
$\sim_q$, and similarly for a right quotient.

For example, consider $q_1 = aa^* b (cc^* \mapsto c)$, for which
we have
\begin{align*}
  l_{q_1}: \niceFrac{\llbracket aa^*bcc^* \rrbracket}{\sim_{q_1}} \lensbetween{}
  \llbracket aa^*bc \rrbracket
\end{align*}
Take the left quotient of this by $q_0 = aa^* (bb^* \mapsto b) cc^*$
This gives a lens
\begin{align*}
  l': \niceFrac{\llbracket aa^*bb^*cc^*\rrbracket}{\sim} \lensbetween{}
  \llbracket aa^*bc \rrbracket
\end{align*}
where
\begin{align*}
  a^n b^{m_1} c^{k_1} \sim a^n b^{m_2} c^{k_2}
\end{align*}
for all $n, m_1, m_2, k_1, k_2 \geq 1$.
This is different from $\sim_{q_0}$.

\begin{definition}
  The expressible quotient lenses are defined inductively
  as follows.
  \begin{itemize}
  \item Every bijective lens
    $\niceFrac{\llbracket r_1 \rrbracket}{=} \lensbetween{}
    \niceFrac{\llbracket r_2 \rrbracket}{=}$
    is expressible.
  \item $l_q$ and $l_q^{-1}$ is expressible for every QRE $q$.
  \item $\lquot \; q \; l$ is expressible if it is well-defined
    and $l$ is expressible.
  \item $\rquot \; l \; q$ is expressible if it is well-defined
    and $l$ is expressible.
  \item If $l = l_1; l_2$ is well-defined, it is expressible.
  \end{itemize}
\end{definition}
Expressible lenses are the complete set of quotient lenses that can
be expressed using the language.
We want to check if all expressible lenses can be normalized to a
synthesizable lens.

\begin{theorem}
  \label{thm:bijective-syn}
  Every bijective lens
  $\llbracket r_1 \rrbracket \lensbetween{l} \llbracket r_2
  \rrbracket$
  is equal to the lens
  \begin{align*}
    W(r_1) \lensbetween{\mathit{id}_{\llbracket r_1 \rrbracket}} K(r_1)
    \lensbetween{l}
    K(r_2) \lensbetween{\mathit{id}_{\llbracket r_2 \rrbracket}} W(r_2)
  \end{align*}
  Hence every bijective lens is synthesizable.
\end{theorem}


\begin{theorem}
  \label{thm:kernel-rep}
  For every QRE $q$, $K(q) = \llbracket r \rrbracket$ for some
  regular expression $r$.
\end{theorem}
\begin{proof}
  By induction on QREs.
\end{proof}

\begin{theorem}
  \label{thm:basic-qre-syn}
  For any QRE $q$, $l_q$ and $l_q^{-1}$ are synthesizable.
\end{theorem}
\begin{proof}
  We just consider the two cases.

  For $l_q$, by Theorem~\ref{thm:kernel-rep},
  there exists RE $r$ such that
  $K(q) = \llbracket r \rrbracket = K(r)$.
  So $l$ is equal to 
  \begin{align*}
    \niceFrac{W(q)}{\sim_{q}} \lensbetween{l_q}
    K(q) \lensbetween{\mathit{id}_{\llbracket r \rrbracket}} K(r)
    \lensbetween{\mathit{id}_{\llbracket r \rrbracket}}
    \niceFrac{W(r)}{=}
  \end{align*}
  The identity lens $\mathit{id}_{\llbracket r \rrbracket}$ is
  clearly bijective.
  
  For $l_q^{-1}$, by Theorem~\ref{thm:kernel-rep},
  there exists RE $r$ such that
  $K(q) = \llbracket r \rrbracket = K(r)$.
  Then $l$ is equal to 
  \begin{align*}
    \niceFrac{W(r)}{=}
    \lensbetween{\mathit{id}_{\llbracket r \rrbracket}} K(r)
    \lensbetween{\mathit{id}_{\llbracket r \rrbracket}}
    K(q) \lensbetween{l_q} \niceFrac{W(q)}{\sim_q}
  \end{align*}
\end{proof}
The normalization algorithm is as follows. Given an expressible lens:
\begin{itemize}
\item If the expressible lens $l$ is $l_q$ or $l_q^{-1}$ for some QRE
  $q$, synthesize as per Theorem~\ref{thm:basic-qre-syn}.
\item If the expressible lens is $\lquot \; q \; l$ for some QRE $q$
  and expressible lens $l$, first normalize $l$
  to $l: \niceFrac{W(q_1^l)}{\sim_{q_1^l}} \lensbetween{}
  \niceFrac{W(q_2^l)}{\sim_{q_2^l}}$. Then consider the
  cases on $q$:
  \begin{itemize}
  \item If $q$ is a regular expression $r$, then
    $c_q = \mathit{id}_{r}$ and there is nothing to do.
  \item If $q = r \mapsto s$, then
    \begin{align*}
      l: \niceFrac{\llbracket s \rrbracket}{=} \lensbetween{}
      W(q_2^l)/\sim_{q_2^l}
    \end{align*}
    since the only quotient relation on $s$ is equality.
    Then we can simply quotient with $c_q$ on the left to get
    a synthesizable lens.
    
  \item If $q = \perm(q_1, \ldots, q_n) \with q_0$,
  \item If $q = q_0^*$,
    normalize $l_{q_0}$ to
    $l_{q_0}: \niceFrac{W(q_0')}{\sim_{q_0'}} \lensbetween{}
    \niceFrac{W(q_0'')}{\sim_{q_0''}}$. 
    The left-quotient is only well-defined if
    $K(q) = K(q_0)^* = W(q_0'')^* = W(q_1^l)$, and
    
  \item If $q = q_1 \cdot q_2$,
  \item If $q = q_1 | q_2$, 
  \end{itemize}
\item 
\end{itemize}


\end{document}
