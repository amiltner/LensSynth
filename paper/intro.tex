
\section{Introduction}
\begin{figure}
\begin{tabular}{l@{\ }l@{\ }c@{\ }l@{\ }r}

% REGEX
(Regexs)& \Regex{} & \GEq{} & \UserDefined{} & variable \\
& & & \GBar{} s $\in \Sigma^*$ & base \\
& & & \GBar{} \Regex{}* & star \\
& & & \GBar{} $\Regex_1 \Regex_2$ & concat \\
& & & \GBar{} $\Regex_1 | \Regex_2$ & or \\
\end{tabular}
\caption{Regex Syntax}
\label{fig:refn-subgrammars}
\end{figure}

\begin{figure}
\[
\begin{array}{lcl}
\LanguageOf{\String} &=& \{\String\}\\
\LanguageOf{\emptyset} &=& \{\}\\
\LanguageOf{\RegexConcat{\Regex_1}{\Regex_2}} &=&
\{\StringConcat{\String_1}{\String_2} \SuchThat
\String_1\in\LanguageOf{\Regex_1} \BooleanAnd \String_2\in\LanguageOf{\Regex_2}\}\\
\LanguageOf{\RegexOr{\Regex_1}{\Regex_2}} &=&
\{\String \SuchThat
\String\in\LanguageOf{\Regex_1} \BooleanOr \String\in\LanguageOf{\Regex_2}\}\\
\LanguageOf{\StarOf{\Regex}} &=&
\{\String_1\Concat\ldots\Concat\String_n \SuchThat
n\in\Nats \wedge \String_i\in\LanguageOf{\Regex}\}
\end{array}
\]
\caption{Regex Semantics}
\label{fig:regex-semantics}
\end{figure}



\begin{figure}

let $\Delta$ be the set of user defined data types.
let $\Sigma^*$ be the set of words over the alphabet $\Sigma$\\

\begin{tabular}{l@{\ }l@{\ }c@{\ }l@{\ }r}

% REGEX
(Lenses)& \Lens{} & \GEq{} & $const(s_1 \in \Sigma^*,s_2 \in \Sigma^*)$ & const \\
& & & \GBar{} $\IdentityLens$ & identity\\
& & & \GBar{} $\IterateLens{\Lens}$ & iterate \\
& & & \GBar{} $\ConcatLens{\Lens_1}{\Lens_2}$ & concat \\
& & & \GBar{} $\SwapLens{\Lens_1}{\Lens_2}$ & swap\\
& & & \GBar{} $\OrLens{\Lens_1}{\Lens_2}$ & or\\
& & & \GBar{} $\ComposeLens{\Lens_1}{\Lens_2}$ & compose\\
\end{tabular}
\caption{Lens Syntax}
\label{fig:lens-syntax}
\end{figure}

\begin{figure}
\centering
\begin{mathpar}
\inferrule[\ConstantLensRule{}]
{
\String_1 \in \StarOf{\Sigma}\\
\String_2 \in \StarOf{\Sigma}
}
{
\ConstLens{\String_1}{\String_2} \OfType \String_1 \Leftrightarrow \String_2 \HasSemantics \lambda \String. \String_2 , \lambda \String. \String_1
}

\inferrule[\IdentityLensRule{}]
{
}
{
\IdentityLens \OfType \Regex \Leftrightarrow \Regex \HasSemantics \lambda \String.\String, \lambda \String . \String
}

\inferrule[\IterateLensRule{}]
{
\Lens \OfType \Regex \Leftrightarrow \RegexAlt \HasSemantics \PutRight , \PutLeft \\
\UnambigItOf{\LanguageOf{\Regex}}\\
\UnambigItOf{\LanguageOf{\RegexAlt}}
}
{
\IterateLens{\Lens} \OfType \StarOf{\Regex} \Leftrightarrow \StarOf{\RegexAlt} \HasSemantics\\\\
\lambda \String.\LetWhereIn{\String_1\Concat\ldots\Concat\String_n}{\String}{\String_i\in\LanguageOf{\Regex}} (\PutRight\Apply\String_1)\Concat\ldots\Concat(\PutRight\Apply\String_n),\\\\
\lambda \String.\LetWhereIn{\String_1\Concat\ldots\Concat\String_n}{\String}{\String_i\in\LanguageOf{\RegexAlt}} (\PutLeft\Apply\String_1)\Concat\ldots\Concat(\PutLeft\Apply\String_n)
}

\inferrule[\ConcatLensRule{}]
{
\Lens_1 \OfType \Regex_1 \Leftrightarrow \RegexAlt_1 \HasSemantics \PutRight_1, \PutLeft_1\\
\Lens_2 \OfType \Regex_2 \Leftrightarrow \RegexAlt_2 \HasSemantics \PutRight_2, \PutLeft_2\\\\
\UnambigConcatOf{\LanguageOf{\Regex_1}}{\LanguageOf{\Regex_2}}\\
\UnambigConcatOf{\LanguageOf{\RegexAlt_1}}{\LanguageOf{\RegexAlt_2}}
}
{
\ConcatLens{\Lens_1}{\Lens_2} \OfType \Regex_1\Regex_2 \Leftrightarrow \RegexAlt_1\RegexAlt_2 \HasSemantics\\\\
\lambda \String.\LetWhereIn{\String_1\Concat\String_2}{\String}{\String_i\in\LanguageOf{\Regex_i}} (\PutRight_1\Apply\String_1)\Concat(\PutRight_2\Apply\String_2),\\\\
\lambda \String.\LetWhereIn{\String_1\Concat\String_2}{\String}{\String_i\in\LanguageOf{\RegexAlt_i}} (\PutLeft_1\Apply\String_1)\Concat(\PutLeft_2\Apply\String_2)
}

\inferrule[\SwapLensRule{}]
{
\Lens_1 \OfType \Regex_1 \Leftrightarrow \RegexAlt_1 \HasSemantics \PutRight_1,\PutLeft_2\\
\Lens_2 \OfType \Regex_2 \Leftrightarrow \RegexAlt_2 \HasSemantics \PutRight_1,\PutLeft_2\\\\
\UnambigConcatOf{\LanguageOf{\Regex_1}}{\LanguageOf{\Regex_2}}\\
\UnambigConcatOf{\LanguageOf{\RegexAlt_2}}{\LanguageOf{\RegexAlt_1}}
}
{
\SwapLens{\Lens_1}{\Lens_2} \OfType \Regex_1\Regex_2 \Leftrightarrow \RegexAlt_2\RegexAlt_1 \HasSemantics\\\\
\lambda \String.\LetWhereIn{\String_1\Concat\String_2}{\String}{\String_i\in\LanguageOf{\Regex_i}} (\PutRight_2\Apply\String_2)\Concat(\PutRight_1\Apply\String_1),\\\\
\lambda \String.\LetWhereIn{\String_1\Concat\String_2}{\String}{\String_i\in\LanguageOf{\RegexAlt_i}} (\PutLeft_2\Apply\String_2)\Concat(\PutLeft_1\Apply\String_1)
}

\inferrule[\OrLensRule{}]
{
\Lens_1 \OfType \Regex_1 \Leftrightarrow \RegexAlt_1 \HasSemantics \PutRight_1,\PutLeft_1\\
\Lens_2 \OfType \Regex_2 \Leftrightarrow \RegexAlt_2 \HasSemantics \PutRight_2,\PutLeft_2\\\\
\UnambigOrOf{\LanguageOf{\Regex_1}}{\LanguageOf{\Regex_2}}\\ \UnambigOrOf{\LanguageOf{\RegexAlt_1}}{\LanguageOf{\RegexAlt_2}}
}
{
\OrLens{\Lens_1}{\Lens_2} \OfType \Regex_1 | \RegexAlt_1 \Leftrightarrow \Regex_2 | \RegexAlt_2 \HasSemantics\\\\
\lambda \String.\{\PutRight_1(\String) \text{ if $\String\in\LanguageOf{\Regex_1}$ }, \PutRight_2(\String) \text{ if $\String\in\LanguageOf{\Regex_2}$ }\},\\\\
\lambda \String.\{\PutLeft_1(\String) \text{ if $\String\in\LanguageOf{\RegexAlt_1}$ }, \PutLeft_2(\String) \text{ if $\String\in\LanguageOf{\RegexAlt_2}$ }\}
}

\inferrule[\ComposeLensRule{}]
{
\Lens_1 \OfType \Regex_1 \Leftrightarrow \Regex_2 \HasSemantics \PutRight_1,\PutLeft_1\\
\Lens_2 \OfType \Regex_2 \Leftrightarrow \Regex_3 \HasSemantics \PutRight_2,\PutLeft_2\\
}
{
\ComposeLens{\Lens_2}{\Lens_1} \OfType \Regex_1 \Leftrightarrow \Regex_3 \HasSemantics
\PutRight_2\Compose\PutRight_1,\PutLeft_2\Compose\PutLeft_1
}

\inferrule[\RetypeLensRule{}]
{
\Lens \OfType \Regex_1 \Leftrightarrow \Regex_2 \HasSemantics \PutRight,\PutLeft\\
\Regex_1 \equiv \Regex_1'\\
\Regex_2 \equiv \Regex_2'
}
{
\Lens \OfType \Regex_1' \Leftrightarrow \Regex_2' \HasSemantics \PutRight,\PutLeft
}
\end{mathpar}

\caption{Lens Semantics and Typing\bcp{This reads OK, but I still don't
    understand why it wouldn't work to define the semantics separately (and
    then show that every {\em well-typed} lens denotes a bijection).  Seems
    a bit easier for readers that way here, and significantly easier once we
    get to DNF lenses...}}
\label{fig:lens-semantics}
\end{figure}

\begin{figure}
\[
\begin{array}{rcl}
\SemanticsOf{const(\String_1,\String_2)} &=& \SetOf{(\String_1,\String_2)}\\

\SemanticsOf{\IdentityLens} &=& \SetOf{(\String,\String)}\\

\SemanticsOf{\IterateLens{\Lens}} &=& \SetOf{(\String_1\Concat\ldots\Concat\String_n,
\StringAlt_1\Concat\ldots\Concat\StringAlt_n)\SuchThat
(\String_i,\StringAlt_i)\in\SemanticsOf{\Lens}}\\

\SemanticsOf{\ConcatLens{\Lens_1}{\Lens_2}} &=&
\SetOf{(\String_1\Concat\String_2,\StringAlt_1\Concat\StringAlt_2)\SuchThat\\
& & \hspace*{2em}(\String_1,\StringAlt_1)\in\SemanticsOf{\Lens_1}\BooleanAnd
(\String_2,\StringAlt_2)\in\SemanticsOf{\Lens_2}}\\

\SemanticsOf{\SwapLens{\Lens_1}{\Lens_2}} &=&
\SetOf{(\String_1\Concat\String_2,\StringAlt_2\Concat\StringAlt_1)\SuchThat\\
& & \hspace*{2em}(\String_1,\StringAlt_1)\in\SemanticsOf{\Lens_1}\BooleanAnd
(\String_2,\StringAlt_2)\in\SemanticsOf{\Lens_2}}\\

\SemanticsOf{\OrLens{\Lens_1}{\Lens_2}} &=&
\SetOf{(\String,\StringAlt)
\SuchThat(\String,\StringAlt)\in\SemanticsOf{\Lens_1}
\BooleanOr(\String,\StringAlt)\in\SemanticsOf{\Lens_2}}\\

\SemanticsOf{\ComposeLens{\Lens_1}{\Lens_2}} &=&
\SetOf{(\String_1,\String_3)\SuchThat\exists\String_2\\
& & \hspace*{2em}(\String_1,\String_2)\in\SemanticsOf{\Lens_1}\BooleanAnd
(\String_2,\String_3)\in\SemanticsOf{\Lens_2}}
\end{array}
\]
\caption{Lens Semantics}
\label{fig:lens-semantics}
\end{figure}

We have regular expressions as normal regular expressions.
We expand it with having user defined data types as well.
We have lenses, defined in Figure~\ref{fig:lens-syntax}, typed as in Figure~\ref{fig:lens-typing}.
These lenses have a semantic interpretation as a relation, given via Figure~\ref{fig:lens-semantics}.
\begin{theorem}
\label{thm:lens-bij-fcn}
If $\Delta \vdash \Lens : \Regex_1 \Leftrightarrow \Regex_2$,
then for all $x \in \LanguageOf{\Delta}{\Regex_1}$, there exists a unique $y \in \LanguageOf{\Delta}{\Regex_2}$ such that $\denot{\Lens}(x,y)$.
This defines a bijective function called $\Lens.putr : \LanguageOf{\Delta}{\Regex_1} \rightarrow \LanguageOf{\Delta}{\Regex_2}$,
whose inverse is called $\Lens.putl$.
\end{theorem}
We would like to be able to synthesize these lenses automatically, given a
specification as two regular expressions, an a set of values that are
mapped to each other.
