\section{Proofs}

\subsection{Soundness}
We say that dnf lenses are \textit{sound} if, there is a dnf lens between two
dnf regular expressions, then between any two regular expressions equivalent
to the two dnf regular expressions, there is a lens between those regular
expressions such that the lens and dnf lens have the same semantics.

\begin{definition}[repregex]
We define a representative regex for a dnf regex as follows:
\begin{itemize}
\item $repregex([\Clause])=repregex(\Clause)$
\item $repregex([\Clause_1 ; \ldots ; \Clause_n])$ =\\ $repregex(\Clause_1)| repregex([\Clause_2 ; \ldots ; \Clause_n]))$
\item $regregex([\String]) = \String$
\item $repregex([\String_1 ; \Atom_1 ; \ldots ; \Atom_n ; \String_{n+1}])$ =\\ $\String_1 (repregex(\Atom_1) repregex([\String_2 ; \Atom_2 ; \ldots ; \String_{n+1}]))$
\item $repregex(\UserDefined) = \UserDefined$
\item $repregex(\Star{\DNFRegex}) = \Star{repregex(\DNFRegex)}$
\end{itemize}
\end{definition}

\begin{lemma}[Equivalence of repregex]
$\LanguageOf{\DNFRegex}=\LanguageOf{repregex(\DNFRegex)}$,
$\LanguageOf{\Clause} = \LanguageOf{repregex(\Clause)}$,
and $\LanguageOf{\Atom} = \LanguageOf{repregex(\Atom)}$
\begin{proof}
By induction on the structure of \DNFRegex, \Clause, and \Atom
\end{proof}
\end{lemma}

\begin{definition}[Adjacent Swapping Permutation]
Let $\sigma_{i} : [0,n] \rightarrowtail [0,n]$ be the permutation, where
$\sigma_{i}(i) = i+1$, $\sigma_{i}(i+1) = i$, $\sigma_{i,j}(k\neq i,i+1) = k$
\end{definition}

\begin{lemma}[Expressibility of Adjacent Swapping Permutation]
Let $\sigma_{i}$ be an adjacent element swapping permutation.  The language of
lenses can express $([(\String_1,\String_1) ; \IdentityLens ; \ldots ; \IdentityLens ; (\String_n,\String_n)], \sigma_{i})$.
\begin{proof}
Consider the regular expressions $repregex([\String_1;\Atom_1; \ldots;\String_i])$ $(repregex(\Atom_i) (\String_{i+1} repregex(\Atom_{i+1})))$ $repregex([\String_{i+2} ; \ldots ; \Atom_{n} ; \String_{n+1}])$ and 
$repregex([\String_{1,2};\Atom_{1,2}; \ldots;\String_{i,1}])$ $((\Atom_{i+1,1} \String_{i+1,1})\Atom_{i,1})$ $repregex([\String_{i+2,1} ; \ldots ; \Atom_{n} ; \String_{n+1}])$
Consider the lens between them\\ $\ConcatLens{\ConcatLens{\IdentityLens}{\SwapLens{\IdentityLens}{\SwapLens{\IdentityLens}{\IdentityLens}}}}{\IdentityLens}$
By inspection, this lens is equivalent to the adjacent swapping permutation.
\end{proof}
\end{lemma}

\begin{lemma}[Expressibility of Permutation]
The language of lenses can express $([(\String_1,\String_1) ; \IdentityLens ; \ldots ; \IdentityLens ; (\String_n,\String_n)], \sigma)$
for any permutation $\sigma$.
\begin{proof}
Let $\sigma$ be a permutation.
Consider the clause lens $([(\String_1,\String_1) ; \IdentityLens ; \ldots ; \IdentityLens ; (\String_n,\String_n)], \sigma_{i})$.
From algebra, we know that the group of permutations is generated by all
adjacent swaps $\sigma_i = (i,i+1)$.
So there exists an adjacency swap decomposition of $\sigma = \sigma_{i_1}\ldots\sigma_{i_m}$.
Consider the dnf lens $([(\String_1,\String_1) ; \IdentityLens ; \ldots ; \IdentityLens ; (\String_n,\String_n)], \sigma_{i_j})$ for each $\sigma_{i_j}$.
By the above lemma, there exists a $\Lens_j$ for each of these adjacency swaps.
Consider the lens $\Lens = \ComposeLens{\Lens_{i_1}}{\ComposeLens{\Lens_{i_2}}{\ldots \ComposeLens{}{\Lens_{i_m}}}}$
By the semantics, they are the same.

\end{proof}
\end{lemma}

\begin{theorem}[Soundness]
Let \Regex{} and \RegexAlt{} be two regular expressions, and \DNFRegex{} and \DNFRegexAlt{} be two dnf regular expressions.
If \LanguageOf{\Regex{}} = \LanguageOf{\DNFRegex{}} and \LanguageOf{\RegexAlt} = \LanguageOf{\DNFRegexAlt{}},
then if there exists a dnf lens $\DNFLens : \DNFRegex \Leftrightarrow \DNFRegexAlt$,
then there exists a lens $\Lens : \Regex \Leftrightarrow \RegexAlt$ such that
$\DNFLens.putr = \Lens.putr$.
\begin{proof}
By induction on the typing \DNFLens{}.
\begin{enumerate}
\item[DNF Lens Intro] Let $\Delta \vdash ([\ClauseLens_1 ; \ldots ; \ClauseLens_n],\sigma) : [\Clause_{1,1}, \ldots, \Clause_{n,1}] \Leftrightarrow [\Clause_{\sigma(1),2}, \ldots, \Clause_{\sigma(n),2}]$.
This comes from the derivations that $\Delta \vdash \ClauseLens_i : \Clause_{i,1} \Leftrightarrow \Clause_{i,2}$ for all $i$.
Consider instead the lens $\DNFLens' = \Delta \vdash ([\ClauseLens_1 ; \ldots ; \ClauseLens_n],\sigma_id) : [\Clause_{1,1}, \ldots, \Clause_{n,1}] \Leftrightarrow[\Clause_{1,2}, \ldots, \Clause_{n,2}]$.
By Lemma (TODO: this lemma), these two lenses are semantically equivalent.
By Lemma (TODO: this lemma), $\Delta \vdash repregex(\DNFLens') : repregex(\DNFRegex_1) \Leftrightarrow repregex(\DNFRegex_2')$, with $repregex(\DNFLens)$ semantically equivalent to $\DNFLens$.
So $repregex(\DNFLens')$ is semantically equivalent to $\DNFLens'$, and $DNFLens'$ is semantically equivalent $\DNFLens$, so $repregex(\DNFLens')$ is
semantically equivalent to $\DNFLens$.  Merely adding in a retyping rule at
the end of $repregex(\DNFLens')$ (as they have the same type), completes this case.
\item[Clause Lens Intro] Let $\Delta \vdash ([(\String_{1,1},\String_{1,2}) ; \AtomLens_1 ; \ldots ; \AtomLens_n ; (\String_{n+1,1},\String_{n+1,2})],\sigma) : [\String_{1,1};\Atom_{1,1}; \ldots ; \Atom_{1,n} ; \String_{1,n+1}] \Leftrightarrow [\String_{1,2};\Atom_{\sigma(1),2}; \ldots ; \Atom_{\sigma(n),2} ; \String_{n+1,2}]$.
Consider two lenses, $\Delta \vdash ([(\String_{1,1},\String_{1,2}) ; \AtomLens_1 ; \ldots ; \AtomLens_n ; (\String_{n+1,1},\String_{n+1,2})],\sigma_id) : [\String_{1,1};\Atom_{1,1}; \ldots ; \Atom_{1,n} ; \String_{1,n+1}] \Leftrightarrow [\String_{1,2};\Atom_{1,2}; \ldots ; \Atom_{n,2} ; \String_{n+1,2}]$,
and $\Delta \vdash ([(\String_{1,2},\String_{1,2});identitylens(\Atom_{1,2}); \ldots ; identitylens(\Atom_{n,2}) ; (\String_{n+1,2},\String_{n+1,2})],\sigma) : [\String_{1,2};\Atom_{1,2}; \ldots ; \Atom_{n,2} ; \String_{n+1,2}] \Leftrightarrow [\String_{1,2};\Atom_{\sigma(1),2}; \ldots ; \Atom_{\sigma(n),2} ; \String_{n+1,2}]$.
By Lemma (TODO:), there exists a lens equivalent to the first one, call it $\Lens_{transform}$.
By Lemma (TODO:), there exists a lens equivalent to the second one, call it $\Lens_{\sigma}$.
Consider $\ComposeLens{\Lens_{\sigma}}{\Lens_{transform}}$.  Go through semantics, oh look they are equivalent.

\end{enumerate}
\end{proof}
\end{theorem}

