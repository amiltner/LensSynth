\section{Related Work}
\subsection{FlashFill}
FlashFill is another string transformation program.  FlashFill takes input and
output examples as its only form of specification.  For FlashFill's primary use
case, providing easy transformations of Excel data, input and output examples as
the only specification makes a lot of sense.  Our tool is intended for use for
programmers to use, who already know regular expressions.  Our tool is oriented
towards a situation where a programmer would check the generated code into a
repository.  From requiring this extra information, our tool is able to handle
synthesize complicated functions that go beyond what FlashFill can synthesize.
For example, given 21 examples of extracting the first author, with the names
reversed, FlashFill still gave incorrect outputs for a number of inputs, even
when those inputs are a similar form to the provided examples.

Furthermore, even if FlashFill were to synthesize the correct program, it would
be foolhardy to check the generated code into a repository.  FlashFill does not
fail on inputs it doesn't know how to handle, rather it tries to provide an
output.  This is different from the generated lenses, which will quickly
fail on inputs that do not match it's intended specification.  Furthermore, the
specification of what inputs should be handled are provided by the user, not
inferred from the examples, giving it more reliability. \afm{talk about how it
  did on bibtex example, more specific on what it did wrong, maybe?}

\subsection{FlashExtract}
FlashExtract is a program that takes an input of a string, and a partial
labeling of substrings within that string, to try to label the rest of the
substrings.  FlashExtract was able to label some amounts of the strings, but was
unable to handle reorderings of the data well.  While it would do well in
extracting data when the data came in the same form, for example, when given a
Bibtex file where the fields come in a specific order, the data would do well,
however, FlashExtract was unable to handle when fields were given in different
orders, even if previous examples in that order were given.


%%% Local Variables:
%%% TeX-master: "main"
%%% End: