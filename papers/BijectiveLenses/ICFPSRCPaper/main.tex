\PassOptionsToPackage{usenames,dvipsnames,svgnames,table}{xcolor}
\documentclass[a4paper,twoside]{article}

\listfiles

\usepackage[usenames,dvipsnames,svgnames,table]{xcolor}
\usepackage[margin=1in]{geometry}
\usepackage{amsmath}
\usepackage{mathtools}
\usepackage{bussproofs}
\usepackage{amsthm}
\usepackage{csvsimple}
\usepackage{thmtools,thm-restate}
\usepackage{changepage}
\usepackage{booktabs}
\usepackage{amssymb}
\usepackage[inline]{enumitem}
\usepackage{multirow,bigdelim}
\usepackage{multicol}
\usepackage{siunitx}
\usepackage{listings}
\usepackage{sansmath}
\usepackage{url}
\usepackage{flushend}
\usepackage{microtype}
\usepackage[utf8]{inputenc}
\usepackage{mathpartir}
\usepackage{empheq}
\usepackage{array}
\usepackage{pgfplots}
\usepackage{stmaryrd}
\usepackage{courier}
\usepackage{qtree}
\usepackage[normalem]{ulem}
\usepackage{relsize}
\usepackage{tikz}
\usepackage{algorithm}
\usepackage[noend]{algpseudocode}
\usepackage{graphicx}
\usepackage{subcaption}
\usepackage{textcomp}
\usepackage{tabularx}
\usepackage{stackengine}
\usepackage{caption}
\usepackage{wrapfig}
\usepackage{remreset}

\usepackage{natbib}
\bibliographystyle{abbrvnat}
\setcitestyle{numbers,square}

\usetikzlibrary{
  er,
  matrix,
  shapes,
  arrows,
  positioning,
  fit,
  calc,
  pgfplots.groupplots,
  arrows.meta
}
\tikzset{>={Latex}}

%%%% Hyperlinks – must come late!
%\usepackage[pdftex,%
%            pdfpagelabels,%
%            linkcolor=blue,%
%            citecolor=blue,%
%            filecolor=blue,%
%            urlcolor=blue]
%           {hyperref}

\newcommand{\Regex}{\ensuremath{r}}         % Regular Expression
\newcommand{\UserDefined}{\ensuremath{U}}   % User Defined
\newcommand{\ExampledRegex}{\ensuremath{er}} % Exampled Regex
\newcommand{\Atom}{\ensuremath{a}}          % Atoms
\newcommand{\Clause}{\ensuremath{cl}}       % Clauses
\newcommand{\DNFRegex}{\ensuremath{dr}}         % Regular Expression
\newcommand{\Context}{\ensuremath{\Gamma}}  % Context
\newcommand{\TypeContext}{\ensuremath{\Delta}} %TypeContext
\newcommand{\String}{\ensuremath{s}}        % String
\newcommand{\ExampleNumberList}{\ensuremath{enl}} %Example Number List
\newcommand{\ExampleNumberListList}{\ensuremath{enll}}
\newcommand{\ExampleStringList}{\ensuremath{esl}}
\newcommand{\StringList}{\ensuremath{sl}}
\newcommand{\Natural}{\ensuremath{n}}
\newcommand{\Interleaving}[1]{\ensuremath{interleaving(#1)}}
\newcommand{\Interleave}{\ensuremath{interleave}}
\newcommand{\BinaryInterleave}[2]{\ensuremath{interleave(#1,#2)}}
\newcommand{\NAryInterleave}[2]{\ensuremath{interleave(#1,\ldots,#2)}}
\newcommand{\Combine}{\ensuremath{combine}}
\newcommand{\List}{\ensuremath{l}}
\newcommand{\ValidCombine}[2]{\ensuremath{validcombine(#1,#2)}}
\newcommand{\Parent}[1]{\ensuremath{parent(#1)}}
\newcommand{\Parented}[1]{\ensuremath{parented(#1)}}
\newcommand{\CombineString}[1]{\ensuremath{combine_{\ExampleStringList}(#1)}}
\newcommand{\CombineList}[1]{\ensuremath{combine_{\ExampleNumberListList}(#1)}}
\newcommand{\Length}[1]{\ensuremath{len(#1)}}


\newcommand{\Lens}{\ensuremath{l}}
\newcommand{\AtomLens}{\ensuremath{al}}
\newcommand{\ClauseLens}{\ensuremath{cll}}
\newcommand{\DNFLens}{\ensuremath{dl}}


\newcommand{\Star}[1]{\ensuremath{#1^*}}
\newcommand{\ConcatLens}[2]{\ensuremath{concat(#1,#2)}}
\newcommand{\SwapLens}[2]{\ensuremath{swap(#1,#2)}}
\newcommand{\OrLens}[2]{\ensuremath{or(#1,#2)}}
\newcommand{\IdentityLens}{\ensuremath{identity}}
\newcommand{\IterateLens}[1]{\ensuremath{iterate(#1)}}
\newcommand{\ComposeLens}[2]{\ensuremath{#1 \circ #2}}

% GRAMMAR OPERATORS
\newcommand{\GBar}{\ensuremath{~|~}}
\newcommand{\GIndent}{\hspace{.5in}}
\newcommand{\GEq}{\ensuremath{::=~}}  
\newcommand{\GEmp}{\ensuremath{\cdot}}



\clubpenalty = 10000
\widowpenalty = 10000
\displaywidowpenalty = 10000

%\setlength{\belowcaptionskip}{-5pt}
%\setlength{\textfloatsep}{15pt}

% Creates a display mode for code in sans serif font
\lstnewenvironment{sflisting}[1][]
  {\lstset{%
    mathescape,
    basicstyle=\small\sffamily,
    aboveskip=5pt,
    belowskip=5pt,
    columns=flexible,
    frame=,
    xleftmargin=1em,#1}\sansmath}
  {}
% end

% Macros
\newcommand{\NameOf}[1]{\CF{#1}}

\begin{document}

\special{papersize=8.5in,11in}
\setlength{\pdfpageheight}{\paperheight}
\setlength{\pdfpagewidth}{\paperwidth}

\author{
  \normalsize Name: Anders Miltner\\
  \normalsize Affiliation: Princeton University\\
  \normalsize Email: amiltner@cs.princeton.edu\\
  \normalsize Address: 35 Olden Street, Princeton NJ 08540\\
  \normalsize Advisor: David Walker\\
  \normalsize Category: Graduate\\
  \normalsize ACM Student ID: 7351631
}

\title{\Large Synthesizing Bijective Lenses}

\maketitle

\paragraph*{Motivation}
Companies are increasingly processing data in ad-hoc formats for customer
analytics, for understanding error logs, and more.
Special programs are often needed to extract the
underlying data contained in these ad-hoc formats oftentimes need special
programs to extract the underlying data.  When data needs to be synchronized
accross multiple ad-hoc formats, programs need to be able to both read data from
the first format, and add that data into the second, and read data in the second
format, and add that data to the first.  Boomerang~\cite{boomerang} is an approach
to this problem:  Boomerang programs use a single expression to express
the conversions from the first source to the second and from the second source
back to the first.  Furthermore, Boomerang provides guarantees about the
round-trip behavior of the data sources.  Unfortunately, providing these
guarantees causes difficulties: Boomerang programs are typically quite difficult
to program in!  Boomerang has a complex type system which requires the user to
think about unambiguity of data formats while also thinking about the fiddly
details of data conversions.  To reduce the difficulty in programming in
Boomerang, we create a tool which synthesizes Boomerang programs, \Optician{}.
\Optician{} takes in data format
descriptors and pairs of input/output examples, and outputs a Boomerang program
mapping between those data formats which respects the examples.

\paragraph*{Background and Related Work}
Boomerang is a certain instance of a class of programming languages:
bidirectional programming languages.  A survey by Czarnecki et al. provides a (slightly dated)
summary of these languages~\cite{DBLP:conf/icmt/CzarneckiFHLST09}.  Furthermore, there have recently been
theoretical approaches to bidirectional languages in general~\cite{DBLP:journals/chinaf/FischerHP15}.

In Boomerang,
if $\Lens$ is a lens, and $\Regex$ and $\RegexAlt$ are regular expressions, then
the typing derivation $\Lens \OfType \Regex \Leftrightarrow \RegexAlt$ means
that the lens $\Lens$ maps between the languages defined by regular expressions
$\Regex$ and $\RegexAlt$.  We focus on a subset of lenses, \emph{bijective
  lenses}.  This subset requires stricter round trip properties than normal
lenses; in particular the two functions defined by the lens must be inverses.

A large amount of work has been put into synthesizing programs by
examples~\cite{tds-pldi,le-pldi-2014,Singh:blinkfill,gulwani-popl-2014,morpheus}. 
Some of this work focuses on data transformation and processing.  However, this
previous data processing work has focused on unidirectional transformations.
Furthermore, most work in synthesizing data transformers rely only on examples
to guide synthesis.  This gives these synthesis tools the complex job of both
understanding the shape of the data as well as synthesizing the program that
transforms the data.  We, however, ask users to provide format descriptions to
our algorithm, and apply type-directed synthesis to these format descriptions.
This approach allows our algorithm to synthesize much more complex
transformations than existing tools.

\paragraph*{Approach and Uniqueness}
From two regular expressions, $\Regex$ and $\RegexAlt$ and a set of examples, we
desire to synthesize a 
lens $\Lens$ such that $\Lens \OfType \Regex \Leftrightarrow \RegexAlt$.
Furthermore, the two functions defined by $\Lens$ should map between the two
strings in each pair of examples.
Because we have the pair of regular expressions forming the type, we preform
type-directed synthesis: synthesis which enumerates only well-typed terms.
Unfortunately, this is difficult to do in Boomerang, as Boomerang has many
equivalences on its types (as every regular expression is equivalent to an
infinite number of other regular expressions), and a composition operator.
Using a na\"ive type-directed synthesis strategy on lenses is very inefficient,
as the above aspects of the language creates a large amount of nondeterminism in
the search through the typing derivation.

To resolve these issues, we have introduced alternative languages, DNF lenses
and DNF regular expressions.  Like regular expressions, DNF regular expressions
describe languages.  Like lenses, DNF lenses encode two functions.  Furthermore,
like lenses, DNF lenses are typed by pairs of expressions.  In the DNF lens case
a derivation $\DNFLens \OfType \DNFRegex \Leftrightarrow \DNFRegexAlt$ means
that a DNF lens provides functions between the languages of $\DNFRegex$ and $\DNFRegexAlt$.
We also prove the soundness and completeness of DNF lenses to lenses.

DNF lenses are better suited to synthesis than lenses for a number of reasons.
DNF lenses contain no composition operator, there are fewer equivalences between
the types (each DNF regular expression is equivalent to fewer other
expressions than regular expressions), and equivalences between types can only
be processed at certain areas of the typing derivation.  This reduces the
nondeterminism from the search for a well-typed lens, making the problem permit
tractable solutions.

\begin{figure}
  \centering
  \begin{tikzpicture}[auto,node distance=1.5cm]
    \node[text width=1.5cm,minimum height=.6cm,align=center,draw,rectangle] (todnfregex) {\ToDNFRegex{}};
    
    \node[align=right, anchor=east] (regex1) [left = .6cm of todnfregex.north west]{\Regex{}};
    \node[align=right, anchor=east] (regex2) [left = .6cm of todnfregex.south west]{\RegexAlt{}};
    \node[align=right, anchor=east] (exs) [below = .2cm of regex2]{ \Examples{} };
    
    \node[align=center] (dnfregex1) [right = .4cm of todnfregex.north east]{\DNFRegex{}};
    \node[align=center] (dnfregex2) [right = .4cm of todnfregex.south east]{\DNFRegexAlt{}};

    \node[text width=3.0cm,minimum height=.6cm,align=center,draw,rectangle] [right = 1.45cm of todnfregex.east] (synthdnflens) {\SynthDNFLens{}};
    \node[align=center] [above = .7cm of synthdnflens] (optician) {\Optician{}};
    
    \node[align=center] [right = .4cm of synthdnflens] (dnflens) {\DNFLens{}};
    
    \node[text width=1.5cm,minimum height=.6cm,align=center,draw,rectangle] [right = .4cm of dnflens] (tolens) {\ToLens{}};
    
    \node[align=center] [right = .6cm of tolens] (lens) {\Lens{}};
    
    
    \path[->] (regex1.east) edge (todnfregex.north west);
    \path[->] (regex2.east) edge (todnfregex.south west);
    
    \path[->] (todnfregex.north east) edge (dnfregex1.west);
    \path[->] (todnfregex.south east) edge (dnfregex2.west);
    
    \path[->] (dnfregex1.east) edge (synthdnflens.north west);
    \path[->] (dnfregex2.east) edge (synthdnflens.south west);
    
    \path[->] (synthdnflens) edge (dnflens);
    
    \path[->] (dnflens) edge (tolens);
    
    \path[->] (tolens) edge (lens);
    \draw[->] ($(exs.east)+(-3pt,0)$) -| node(exsedge) {} (synthdnflens);
    \node[fit={($(todnfregex.west)+(-4pt,0)$) ($(tolens.east)+(4pt,0)$) (exsedge) (dnfregex1) (optician) (dnfregex2)},draw] (surrounding) {};
    % Now place a relation (ID=rel1)
    %\node[text width=2cm,align=center,draw, rectangle] (sketch-gen) [right = .75cm of spec] {\TypeProp{}};
    %\node (below-gen) [below=.5cm of sketch-gen] {};
    %\node[text width=2cm,align=center,draw, rectangle] (sketch-compl)
    %     [right = .25cm of sketch-gen] {\RigidSynth{}};
    %\node (below-compl) [below=.5cm of sketch-compl] {};
    %\node[align=center] (lens) [right = .75cm of sketch-compl] {Lens}; 
    %% Draw an edge between rel1 and node1; rel1 and node2
    %\path[->] (spec) edge node (start-alg) {} (sketch-gen);
    %\path[->] (sketch-gen) edge node(middle) {} (sketch-compl);
    %\path[->] (sketch-compl) edge node[near start](success) {\Success{}} (lens);
    %\draw[<-] (sketch-gen.south) -- +(0,-.5) -| node[above left](failure){\Failure{}} (sketch-compl.south);

    %\node (synth-name) [above=.5cm of middle] {\Optician{}};
    %
    %\node[fit=(sketch-gen) (sketch-compl) (start-alg) (synth-name) (failure) (success) ,draw] (surrounding) {};
  \end{tikzpicture}
  \caption{Schematic Diagram for \Optician{}.  Regular expressions, \Regex{} and
    \RegexAlt{}, and examples, \Examples{}, are given as input.
    First, the function \ToDNFRegex{} converts \Regex{} and \RegexAlt{} into
    their respective DNF forms, \DNFRegex{} and \DNFRegexAlt{}.
    Next, \SynthDNFLens{} synthesizes a DNF lens, \DNFLens{}, from \Regex{},
    \RegexAlt{}, and \Examples{}.
    Finally, \ToLens{} converts \DNFLens{} into \Lens{}, a lens in Boomerang
    that is equivalent to \DNFLens{}.}
  \label{fig:schematic-diagram-synthesis}
\end{figure}

Figure~\ref{fig:schematic-diagram-synthesis} shows the schematic diagram for
\Optician{}.  Two regular expressions, \Regex{} and \RegexAlt{}, and a set of
examples, \Examples{}, are given as input.  The regular expressions are
converted into DNF regular expressions with $\ToDNFRegex$.  Then, type-directed
synthesis is performed on the input DNF regular expressions and the examples, in
\SynthDNFLens{}.  After \SynthDNFLens{} generates a satisfying DNF lens, it is
converted back into a lens with $\ToRegex$.  We prove the correctness of our
synthesis algorithm using DNF lenses: if there is a lens satisfying the
specification, then the synthesis algorithm will return a lens which satisfies
the specification (though not necessarily the same one).

Our work is unique in that we are the first synthesis algorithm for a
bidirectional language.  While it is common to develop domain specific languages
which permit efficient synthesis algorithms, and also it is common to develop
ways of compiling these languages to existing ones, it is less common to provide
completeness results stating exactly which elements of the existing language are
able to be synthesized.  Furthermore, \Optician{} synthesizes string processing
programs from types as well as examples. 

\paragraph*{Results and Contributions}
As we are the first synthesis algorithm on bidirectional programs, we have no
direct comparisons, but by changing the specifications, we were able to compare
our tool with 2 tools which synthesize string transformers (Flash Fill~\cite{gulwani-popl-2014} and
FlashExtract~\cite{le-pldi-2014}), and found our tool 
was able to outperform them.
Our synthesis algorithm was able to synthesize all 39
of the examples in less than 5 seconds, where the other tools synthesized 3 and
4 within the time limit of 10 minutes.
Furthermore, we found that the optimizations we used were critical in making
this synthesis tractible.  

Our contributions are as follows:
\begin{enumerate}
\item We developed a domain specific language custom built for synthesis.  We
  proved this language equivalent to a higher-level, declarative, bijective
  subset of Boomerang.

\item We develop a synthesis algorithm for this domain specific language, and
  leverage it for the synthesis of Boomerang programs.  We prove this algorithm
  correct.

\item We evaluate our algorithm on a set of 39 benchmarks.  We find that the
  optimizations are critical for synthesis of all benchmarks, and find that we
  outperform existing tools on the types of complex tasks that necessitate the
  use of bidirectional languages.
\end{enumerate}

\paragraph*{My Contributions}
This work was done with the collaboration of 4 other researchers: David Walker,
Kathleen Fisher, Benjamin Pierce, and Steve Zdancewic.  Their insights
were critical to the entire process.  I did all the proofs and coded
\Optician{}, however.

% \category{D.3.1}
% {Programming Languages} 
% {Formal Definitions and Theory}
% [Semantics]
%\ifanon\else
%\category{F.4.1}
%{Mathematical Logic and Formal Languages}
%{Mathematical Logic}
%[Proof Theory]
%\category{I.2.2}
%{Artificial Intelligence}
%{Automatic Pro\-gramming}
%[Program Synthesis]
%
%\terms Languages, Theory
%
%\keywords Functional Programming, Proof Search, Program Synthesis, Type Theory
%\fi


% The bibliography should be embedded for final submission.

\bibliography{local,bcp}

\end{document}



%%% Local Variables:
%%% TeX-master: "main"
%%% End:
