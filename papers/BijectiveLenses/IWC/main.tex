% easychair.tex,v 3.5 2017/03/15

\documentclass{easychair}
%\documentclass[EPiC]{easychair}
%\documentclass[EPiCempty]{easychair}
%\documentclass[debug]{easychair}
%\documentclass[verbose]{easychair}
%\documentclass[notimes]{easychair}
%\documentclass[withtimes]{easychair}
%\documentclass[a4paper]{easychair}
%\documentclass[letterpaper]{easychair}

\usepackage{doc}

%\newif\ifdraft\drafttrue      % Uncomment to turn on comments
\newif\ifdraft\draftfalse   % Uncomment to turn off comments

% use this if you have a long article and want to create an index
% \usepackage{makeidx}

% In order to save space or manage large tables or figures in a
% landcape-like text, you can use the rotating and pdflscape
% packages. Uncomment the desired from the below.
%
% \usepackage{rotating}
% \usepackage{pdflscape}

% Some of our commands for this guide.
%
\newcommand{\easychair}{\textsf{easychair}}
\newcommand{\miktex}{MiK{\TeX}}
\newcommand{\texniccenter}{{\TeX}nicCenter}
\newcommand{\makefile}{\texttt{Makefile}}
\newcommand{\latexeditor}{LEd}

%\makeindex

%% Front Matter
%%
% Regular title as in the article class.
%
%
\title{Confluence in Lens Synthesis}

% Authors are joined by \and. Their affiliations are given by \inst, which indexes
% into the list defined using \institute
%
\author{
Anders Miltner\inst{1}
\and
Kathleen Fisher\inst{2}
\and
Benjamin C. Pierce\inst{3}
\and
David Walker\inst{4}
\and
Steve Zdancewic\inst{5}
}

% Institutes for affiliations are also joined by \and,
\institute{
  Princeton University\\
  \email{amiltner@cs.princeton.edu}
\and
   Tufts University\\
   \email{kfisher@eecs.tufts.edu}\\
\and
   University of Pennsylvania\\
   \email{bcpierce@cis.upenn.edu}\\
\and
   Princeton University\\
  \email{dpw@cs.princeton.edu}
\and
  University of Pennsylvania\\
  \email{stevez@cis.upenn.edu}\\
}

%  \authorrunning{} has to be set for the shorter version of the authors' names;
% otherwise a warning will be rendered in the running heads. When processed by
% EasyChair, this command is mandatory: a document without \authorrunning
% will be rejected by EasyChair

\authorrunning{Miltner, Fisher, Pierce, Walker and Zdancewic}

% \titlerunning{} has to be set to either the main title or its shorter
% version for the running heads. When processed by
% EasyChair, this command is mandatory: a document without \titlerunning
% will be rejected by EasyChair
\titlerunning{Confluence in Lens Synthesis}

\usepackage{amsmath}
\usepackage{mathtools}
\usepackage{bussproofs}
\usepackage{amsthm}
\usepackage{csvsimple}
\usepackage{thmtools,thm-restate}
\usepackage{changepage}
\usepackage{booktabs}
\usepackage{amssymb}
\usepackage[inline]{enumitem}
\usepackage{multirow,bigdelim}
\usepackage{multicol}
\usepackage{siunitx}
\usepackage{listings}
\usepackage{sansmath}
\usepackage{url}
\usepackage{flushend}
\usepackage{microtype}
\usepackage[utf8]{inputenc}
\usepackage{mathpartir}
\usepackage{empheq}
\usepackage{array}
\usepackage{pgfplots}
\usepackage{stmaryrd}
\usepackage{courier}
\usepackage{qtree}
\usepackage[normalem]{ulem}
\usepackage{relsize}
\usepackage{tikz}
\usepackage{algorithm}
\usepackage[noend]{algpseudocode}
\usepackage{graphicx}
\usepackage{subcaption}
\usepackage{textcomp}
\usepackage{tabularx}
\usepackage{stackengine}
\usepackage{caption}
\usepackage{wrapfig}
\usepackage{remreset}
\usepackage{tabulary}

\usetikzlibrary{
  er,
  matrix,
  shapes,
  arrows,
  positioning,
  fit,
  calc,
  pgfplots.groupplots,
  arrows.meta
}
\tikzset{>={Latex}}

\newcommand{\Regex}{\ensuremath{r}}         % Regular Expression
\newcommand{\UserDefined}{\ensuremath{U}}   % User Defined
\newcommand{\ExampledRegex}{\ensuremath{er}} % Exampled Regex
\newcommand{\Atom}{\ensuremath{a}}          % Atoms
\newcommand{\Clause}{\ensuremath{cl}}       % Clauses
\newcommand{\DNFRegex}{\ensuremath{dr}}         % Regular Expression
\newcommand{\Context}{\ensuremath{\Gamma}}  % Context
\newcommand{\TypeContext}{\ensuremath{\Delta}} %TypeContext
\newcommand{\String}{\ensuremath{s}}        % String
\newcommand{\ExampleNumberList}{\ensuremath{enl}} %Example Number List
\newcommand{\ExampleNumberListList}{\ensuremath{enll}}
\newcommand{\ExampleStringList}{\ensuremath{esl}}
\newcommand{\StringList}{\ensuremath{sl}}
\newcommand{\Natural}{\ensuremath{n}}
\newcommand{\Interleaving}[1]{\ensuremath{interleaving(#1)}}
\newcommand{\Interleave}{\ensuremath{interleave}}
\newcommand{\BinaryInterleave}[2]{\ensuremath{interleave(#1,#2)}}
\newcommand{\NAryInterleave}[2]{\ensuremath{interleave(#1,\ldots,#2)}}
\newcommand{\Combine}{\ensuremath{combine}}
\newcommand{\List}{\ensuremath{l}}
\newcommand{\ValidCombine}[2]{\ensuremath{validcombine(#1,#2)}}
\newcommand{\Parent}[1]{\ensuremath{parent(#1)}}
\newcommand{\Parented}[1]{\ensuremath{parented(#1)}}
\newcommand{\CombineString}[1]{\ensuremath{combine_{\ExampleStringList}(#1)}}
\newcommand{\CombineList}[1]{\ensuremath{combine_{\ExampleNumberListList}(#1)}}
\newcommand{\Length}[1]{\ensuremath{len(#1)}}


\newcommand{\Lens}{\ensuremath{l}}
\newcommand{\AtomLens}{\ensuremath{al}}
\newcommand{\ClauseLens}{\ensuremath{cll}}
\newcommand{\DNFLens}{\ensuremath{dl}}


\newcommand{\Star}[1]{\ensuremath{#1^*}}
\newcommand{\ConcatLens}[2]{\ensuremath{concat(#1,#2)}}
\newcommand{\SwapLens}[2]{\ensuremath{swap(#1,#2)}}
\newcommand{\OrLens}[2]{\ensuremath{or(#1,#2)}}
\newcommand{\IdentityLens}{\ensuremath{identity}}
\newcommand{\IterateLens}[1]{\ensuremath{iterate(#1)}}
\newcommand{\ComposeLens}[2]{\ensuremath{#1 \circ #2}}

% GRAMMAR OPERATORS
\newcommand{\GBar}{\ensuremath{~|~}}
\newcommand{\GIndent}{\hspace{.5in}}
\newcommand{\GEq}{\ensuremath{::=~}}  
\newcommand{\GEmp}{\ensuremath{\cdot}}



\begin{document}

\maketitle

\begin{abstract}
  A lens is a program that can be executed both forwards and backwards, from
  input to output and from output back to input again. Domain-specific languages
  for defining lenses have been developed to help users synchronize text files,
  and construct different ``views'' of databases, among other applications.
  Recent research has shown how string lenses can be synthesized from their
  types, which are pairs of regular expressions. However, guaranteeing that we
  can synthesize all possible lenses is quite tricky on these languages, due in
  large part to the many equivalences on regular expressions.

  The proof that all string lenses are synthesizeable involves proving a
  confluence-like property, parameterized by a an additional binary relation
  $R$. We call this property $R$-confluence. In this model, standard confluence
  is the specific case where $R$ is equality. In this paper, we show how
  existing techniques for demonstrating confluence do not work in the domain of
  $R$-confluence, and find that if the rewrite system is $=$-confluent and
  satisfies a commutativity property with $R$, then the system is $R$-confluent.
\end{abstract}

\section{Introduction}

Bidirectional transformations are pervasive in modern software systems, occuring
as database views and view updaters, parsers and pretty-printers, data
synchronization tools, and more. Instead of manually building the functions that
comprise a bidirectional transformation, programmers can build them both ``at
once'' using a bidirectional programming language. Bidirectional programming
languages have been developed for creating view
updaters~\cite{BohannonPierceVaughan}, Linux configuration file
editors~\cite{augeas2}, direct manipulation programming
systems~\cite{bidirectionaleval}, and
more~\cite{DBLP:conf/pepm/KoZH16,DBLP:conf/icfp/HidakaHIKMN10,DBLP:conf/staf/ZhuK0SH15}. Lenses are a
particularly well-behaved class of bidirectional programs, where the underlying
transformations are guaranteed to satisfy a number of ``round-tripping'' laws.
Lens-based bidirectional programming languages often provide round-tripping
guarantees through a set of typing rules; well-typed lens expressions are
guaranteed to satisfy the round-tripping laws.

Optician~\cite{optician}, an extension of Boomerang, makes bidirectional
programming easier by supporting synthesis of bidirectional string
transformations. More specifically, it takes as input two regular expressions
($\Regex$ and $\RegexAlt$, which serve as the type of a Boomerang lens) and a
set of examples specifying input-output behavior, and synthesizes a well-typed
lens between the languages of those regular expressions. For brevity, we will
not provide formal definitions for some aspects of Optician; the interested
reader can find such definitions in the original Optician paper~\cite{optician}.
Furthermore, examples detailing the use of synthesized lenses in practice (which
we also elide for space) can be found in that paper and follow-up
work~\cite{maina+:quotient-synthesis,soptician}.

To indicate a lens $\Lens$ is well typed, and converts between the languages of
$\Regex$ and $\RegexAlt$, we write $\Lens : \Regex \Leftrightarrow \RegexAlt$.
For synthesis, given $\Regex$ and $\RegexAlt$, we must find a lens $\Lens$ such
that $\Lens : \Regex \Leftrightarrow \RegexAlt$. In the context of lens
synthesis, there are three sorts of lens typing rules: syntax-directed rules,
composition, and type equivalence.

For syntax-directed rules, the syntax for the types closely mirrors the syntax
of the expressions. As an example, consider the following rule for
disjunction:
\begin{mathpar}
  \inferrule[\OrLensRule{}]
  {
    \Lens_1 \OfType \Regex_1 \Leftrightarrow \RegexAlt_1\\
    \Lens_2 \OfType \Regex_2 \Leftrightarrow \RegexAlt_2\\\\
    \UnambigOrOf{\Regex_1}{\Regex_2}\\
    \UnambigOrOf{\RegexAlt_1}{\RegexAlt_2}
  }
  {
    \OrLensOf{\Lens_1}{\Lens_2} \OfType
    \Regex_1 \Or \Regex_2
    \Leftrightarrow \RegexAlt_1 \Or \RegexAlt_2
  }
\end{mathpar}
Consider finding a lens of type $\Regex_1 \Or \Regex_2 \Leftrightarrow
\RegexAlt_1 \Or \RegexAlt_2$. With only syntax-directed rules, the only lens
that can be well-typed would be an or lens.

Composition sequentially composes two lenses.
\begin{mathpar}
  \inferrule[\ComposeLensRule{}]
  {
    \Lens_1 \OfType \Regex_1 \Leftrightarrow \Regex_2\\
    \Lens_2 \OfType \Regex_2 \Leftrightarrow \Regex_3\\
  }
  {
    \ComposeLensOf{\Lens_1}{\Lens_2} \OfType \Regex_1 \Leftrightarrow \Regex_3
  }
\end{mathpar}
Composition is difficult in the context of lens synthesis. If trying to
synthesize a composition lens, one has to pull the central regular expression
``out of thin air.''

The last typing rule is type-equivalence. If two regular expressions are
star-semiring equivalent to the type of a lens, those equivalent regular
expressions also serve as the type of the lens.
\begin{mathpar}
  \inferrule[\RewriteRegexLensRule{}]
  {
    \Lens \OfType \Regex \Leftrightarrow \RegexAlt\\
    \Regex \SSREquiv \Regex'\\
    \RegexAlt \SSREquiv \RegexAlt'
  }
  {
    \Lens \OfType \Regex' \Leftrightarrow \RegexAlt'
  }
\end{mathpar}
This rule is difficult in the context of synthesis, as it forces a search
through equivalent regular expressions.  This rule can also be applied at any
point in the derivation, which makes the search even harder.

%\begin{figure}
%  \centering
%  \begin{tikzpicture}[auto,node distance=1.5cm]
%    \node[text width=1.5cm,minimum height=.6cm,align=center,draw,rectangle] (todnfregex) {\ToDNFRegex{}};
%    
%    \node[align=right, anchor=east] (regex1) [left = .6cm of todnfregex.north west]{\Regex{}};
%    \node[align=right, anchor=east] (regex2) [left = .6cm of todnfregex.south west]{\RegexAlt{}};
%    \node[align=right, anchor=east] (exs) [below = .2cm of regex2]{ \Examples{} };
%    
%    \node[align=center] (dnfregex1) [right = .4cm of todnfregex.north east]{\DNFRegex{}};
%    \node[align=center] (dnfregex2) [right = .4cm of todnfregex.south east]{\DNFRegexAlt{}};
%
%    \node[text width=2.9cm,minimum height=.6cm,align=center,draw,rectangle] [right = 1.45cm of todnfregex.east] (synthdnflens) {\SynthDNFLens{}};
%    \node[align=center] [above = .7cm of synthdnflens] (optician) {\Optician{}};
%    
%    \node[align=center] [right = .4cm of synthdnflens] (dnflens) {\DNFLens{}};
%    
%    \node[text width=1.5cm,minimum height=.6cm,align=center,draw,rectangle] [right = .4cm of dnflens] (tolens) {\ToLens{}};
%    
%    \node[align=center] [right = .6cm of tolens] (lens) {\Lens{}};
%    
%    
%    \path[->] (regex1.east) edge (todnfregex.north west);
%    \path[->] (regex2.east) edge (todnfregex.south west);
%    
%    \path[->] (todnfregex.north east) edge (dnfregex1.west);
%    \path[->] (todnfregex.south east) edge (dnfregex2.west);
%    
%    \path[->] (dnfregex1.east) edge (synthdnflens.north west);
%    \path[->] (dnfregex2.east) edge (synthdnflens.south west);
%    
%    \path[->] (synthdnflens) edge (dnflens);
%    
%    \path[->] (dnflens) edge (tolens);
%    
%    \path[->] (tolens) edge (lens);
%    \draw[->] ($(exs.east)+(-3pt,0)$) -| node(exsedge) {} (synthdnflens);
%    \node[fit={($(todnfregex.west)+(-4pt,0)$) ($(tolens.east)+(4pt,0)$) (exsedge) (dnfregex1) (optician) (dnfregex2)},draw] (surrounding) {};
%    % Now place a relation (ID=rel1)
%    %\node[text width=2cm,align=center,draw, rectangle] (sketch-gen) [right = .75cm of spec] {\TypeProp{}};
%    %\node (below-gen) [below=.5cm of sketch-gen] {};
%    %\node[text width=2cm,align=center,draw, rectangle] (sketch-compl)
%    %     [right = .25cm of sketch-gen] {\RigidSynth{}};
%    %\node (below-compl) [below=.5cm of sketch-compl] {};
%    %\node[align=center] (lens) [right = .75cm of sketch-compl] {Lens}; 
%    %% Draw an edge between rel1 and node1; rel1 and node2
%    %\path[->] (spec) edge node (start-alg) {} (sketch-gen);
%    %\path[->] (sketch-gen) edge node(middle) {} (sketch-compl);
%    %\path[->] (sketch-compl) edge node[near start](success) {\Success{}} (lens);
%    %\draw[<-] (sketch-gen.south) -- +(0,-.5) -| node[above left](failure){\Failure{}} (sketch-compl.south);
%
%    %\node (synth-name) [above=.5cm of middle] {\Optician{}};
%    %
%    %\node[fit=(sketch-gen) (sketch-compl) (start-alg) (synth-name) (failure) (success) ,draw] (surrounding) {};
%  \end{tikzpicture}
%  \caption{Schematic Diagram for \Optician{}.  Regular expressions, \Regex{} and
%    \RegexAlt{}, and examples, \Examples{}, are given as input.
%    First, the function \ToDNFRegex{} converts \Regex{} and \RegexAlt{} into
%    their respective DNF forms, \DNFRegex{} and \DNFRegexAlt{}.
%    Next, \SynthDNFLens{} synthesizes a DNF lens, \DNFLens{}, from \Regex{},
%    \RegexAlt{}, and \Examples{}.
%    Finally, \ToLens{} converts \DNFLens{} into \Lens{}, a lens in Boomerang
%    that is equivalent to \DNFLens{}.}
%  \label{fig:schematic-diagram-synthesis}
%\end{figure}

To address the difficulties with composition and type-equivalence rules, we
synthesize lenses in an alternative language of disjunctive normal form (DNF)
lenses. DNF lenses are in pseudonormal form, containing no composition operator,
and so their synthesis never needs to pull regular expressions ``out of thin
air.'' The types of DNF lenses are pairs of regular expressions in a
pseudonormal form, DNF regular expressions. Because DNF regular expressions are
in a pseudonormal form, fewer equivalent regular expressions need to be searched
through.

In our search algorithm, we only search through equivalent regular expressions
once, before processing any syntax directed-rules. We formalize this in the
typing of DNF regular expressions by only permitting the application of
type-equivalence once, after all syntactic rules have been applied. This is
enforced by having two typing judgements: one for the ``rewriteless'' type of
the lens (meaning no type-equivalence rules were applied) and one of the
``full'' type of the lens. If $\DNFLens \OfRewritelessType \DNFRegex
\Leftrightarrow \DNFRegexAlt$, then $\DNFLens$ is a DNF lens of rewriteless type
$\DNFRegex \Leftrightarrow \DNFRegexAlt$. If $\DNFLens \OfType \DNFRegex
\Leftrightarrow \DNFRegexAlt$, then $\DNFLens$ is a DNF lens of full type
$\DNFRegex \Leftrightarrow \DNFRegexAlt$. The following rule is used to get the
full type of a DNF lens from the rewriteless type.
\begin{mathpar}
  \inferrule[\RewriteDNFRegexLensRule{}]
  {
    \DNFRegex' \StarOf{\Rewrite} \DNFRegex\\
    \DNFRegexAlt' \StarOf{\Rewrite} \DNFRegexAlt \\
    \DNFLens \OfRewritelessType \MapsBetweenTypeOf{\DNFRegex}{\DNFRegexAlt}
  }
  {
    \DNFLens \OfType \MapsBetweenTypeOf{\DNFRegex'}{\DNFRegexAlt'}
  }
\end{mathpar}
Note that instead of using directionless equivalences,
\RewriteDNFRegexLensRule{} uses directioned rewrites. The relationship between
the two is formalized by the following theorem:
\begin{theorem}
  \label{thm:thm1}
  If $\Regex \SSREquiv \RegexAlt$, then $\ToDNFRegexOf{\Regex}
  \equiv_{\rightarrow} \ToDNFRegexOf{\RegexAlt}$, where $\ToDNFRegexOf{\Regex}$
  and $\ToDNFRegexOf{\RegexAlt}$ are $\Regex$ and $\RegexAlt$ in DNF form
  (respectively), and $\equiv_{\rightarrow}$ is the reflexive, transitive, and
  symmetric closure of $\rightarrow$.
\end{theorem}

Proving that our search procedure can generate any lens reduces to proving DNF
lenses complete with respect to our standard lens language. In particular, we
wish the prove:
\begin{theorem}
  If $\Lens : \Regex \Leftrightarrow \RegexAlt$, then there exists a DNF lens,
  $\DNFLens$, such that $\DNFLens \OfType \ToDNFRegexOf{\Regex} \Leftrightarrow
  \ToDNFRegexOf{\RegexAlt}$ and the semantics of $\Lens$ and $\DNFLens$
  are equivalent.
\end{theorem}

We prove this propety by induction on the structure of the typing derivation.
Particular difficulty lies in the lens equivalence rule. We begin this case
below:
  
\begin{mathpar}
  \inferrule
  {
    \Lens \OfType \Regex_1 \Leftrightarrow \RegexAlt_1\\
    \Regex_1 \SSREquiv \Regex_2\\
    \RegexAlt_1 \SSREquiv \RegexAlt_2
  }
  {
    \Lens \OfType \Regex_2 \Leftrightarrow \RegexAlt_2
  }
\end{mathpar}

By induction assumption, there exists $\DNFLens \OfType \ToDNFRegexOf{\Regex_1}
\Leftrightarrow \ToDNFRegexOf{\RegexAlt_1}$, where the semantics of $\DNFLens$
are equivalent to those of $\Lens$. By inversion on the derivation of $\DNFLens
\OfType \ToDNFRegexOf{\Regex_1} \Leftrightarrow \ToDNFRegexOf{\RegexAlt_1}$,
there exists $\DNFRegex_1$ and $\DNFRegexAlt_1$ such that:
\begin{mathpar}
  \inferrule
  {
    \ToDNFRegexOf{\Regex_1} \StarOf{\Rewrite} \DNFRegex_1\\
    \ToDNFRegexOf{\RegexAlt_1} \StarOf{\Rewrite} \DNFRegexAlt_1\\
    \DNFLens \OfRewritelessType \DNFRegex_1 \Leftrightarrow \DNFRegexAlt_1
  }
  {
    \DNFLens \OfType \ToDNFRegexOf{\Regex_1} \Leftrightarrow \ToDNFRegexOf{\RegexAlt_1}
  }
\end{mathpar}

To complete this case, we need to find a DNF lens $\DNFLens' \OfType
\ToDNFRegexOf{\Regex_2} \Leftrightarrow \ToDNFRegexOf{\RegexAlt_2}$ with
equivalent semantics to $\Lens$.

We first show that there exist $\DNFRegex_2$ and $\DNFRegexAlt_2$ such that
$\ToDNFRegexOf{\Regex_2} \StarOf{\rightarrow} \DNFRegex_2$ and
$\ToDNFRegexOf{\DNFRegex_1} \StarOf{\rightarrow} \DNFRegex_2$ and
$\ToDNFRegexOf{\RegexAlt_2} \StarOf{\rightarrow} \DNFRegexAlt_2$ and
$\ToDNFRegexOf{\DNFRegexAlt_1} \StarOf{\rightarrow} \DNFRegexAlt_2$. To do this,
we first prove that $\rightarrow$ is confluent. By
Theorem~\ref{thm:thm1}, we know that $\ToDNFRegexOf{\Regex_1} \equiv_\rightarrow
\ToDNFRegexOf{\Regex_2}$, so confluence implies the existance of such a $\DNFRegex_2$ and $\DNFRegexAlt_2$.

\begin{figure}
  \centering
    \includegraphics[scale=.4]{equiv-case.pdf}
    \caption{Diagram constructing a well-typed DNF lens between
      \ToDNFRegexOf{\Regex_2} and \ToDNFRegexOf{\RegexAlt_2}. Single lined
      arrows indicate rewrites ($\rightarrow$), and triple lines indicate
      equivalence ($\equiv_{\rightarrow}$).}
    \label{fig:equiv-case}
\end{figure}

After this, we have $\DNFLens \OfRewritelessType \DNFRegex_1 \Leftrightarrow
\DNFRegexAlt_1$ and $\DNFRegex_1 \StarOf{\rightarrow} \DNFRegex_2$ and
$\DNFRegexAlt_1 \StarOf{\rightarrow} \DNFRegexAlt_2$. If we can prove a
confluence-like property that would show the existance of some $\DNFLens'
\OfRewritelessType \DNFRegex_3 \Leftrightarrow \DNFRegexAlt_3$, where
$\DNFRegex_2 \StarOf{\rightarrow} \DNFRegex_3$ and $\DNFRegexAlt_2
\StarOf{\rightarrow} \DNFRegexAlt_3$, we would be done. This property is
$R$-confluence for a properly chosen $R$ (which we describe in \S\ref{sec:rcf}).
This case is diagrammed in Figure~\ref{fig:equiv-case}.

\section{$R$-Confluence Formulation}
\label{sec:rcf}
Let $S$ be an underlying set, and $\rightarrow$ and $R$ be
binary relations. The rewrite system $(S,\rightarrow)$ is $R$-confluent, if for
all $s_1,s_2 \in S$, if $R(s_1,s_2)$, $s_1 \StarOf{\rightarrow} s_1'$, and $s_2
\StarOf{\rightarrow} s_2'$, then there exist $s_1''$ and $s_2''$ such that
$R(s_1'',s_2'')$, $s_1' \StarOf{\rightarrow} s_1''$, and $s_2
\StarOf{\rightarrow} s_2''$.

In the context of lens synthesis, $S$ is the set of DNF REs,
the rewrites are $\StarOf{\rightarrow}$, and given a DNF lens $\DNFLens$,
$R_\DNFLens(\DNFRegex,\DNFRegexAlt)$ is true if there exists a DNF lens $\DNFLens'$ such
that $\DNFLens' \OfRewritelessType \DNFRegex \Leftrightarrow \DNFRegexAlt$ and
$\DNFLens$ is equivalent to $\DNFLens'$.

Now that $R$-confluence has been formally defined, we can ask ourselves: ``What
is a good approach to proving $R$-confluence?'' One approach is to prove that
our rewrite system is locally confluent, which is equivalent to $=$-confluence
in a terminating system~\cite{huetconf}. Unfortunately, our rewrites are not
terminating, so this approach does not work.

An approach pioneered by Tait and Martin-L\"{o}f~\cite{barendrecht} still works
in non-terminating systems. This approach uses the \emph{diamond property}: A
rewrite system $(S,\rightarrow)$ satisfies the diamond property if $s_1
\rightarrow s_2$ and $s_1 \rightarrow s_3$ implies that there exists $s_4$ such
that $s_2 \rightarrow s_4$, and $s_2 \rightarrow s_4$. If a rewrite system
satisfies the diamond property, then it is also confluent. Unfortunately, this
approach does not work, as the parameterized version of the diamond property
does not imply $R$-confluence.

\section{Proving (S,\StarOf{\rightarrow}) $R$-Confluent}

\begin{figure}
  \centering
    \includegraphics[scale=.4]{cex.pdf}
    \caption{Rewrite system that satisfies the $R$-diamond property, but is not
      $R$-confluent.}
    \label{fig:cex}
\end{figure}

In the Tait and Martin-L\"{o}f approach to proving $(S,\rightarrow)$ confluent,
one must first prove $(S,\rightarrow)$ satisfies the diamond property. Consider
a parameterized property analogous to the diamond property, the
\emph{$R$-diamond property}: A rewrite system $(S,\rightarrow)$ satisfies the
$R$-diamond property if $s_1 \rightarrow s_2$ and $t_1 \rightarrow t_2$ and
$R(s_1,t_1)$ implies that there exists $s_3,t_3$ such that $s_2 \rightarrow s_3$
and $t_2 \rightarrow t_3$ and $R(s_3,t_3)$.

However, satisfying the $R$-diamond property is not sufficient for
$R$-confluence. Consider the simple rewrite system shown in
Figure~\ref{fig:cex}. In this rewrite system, there are 3 elements, $s_1,s_2$,
and $t$. In this setup, $R(s_1,t)$ and $s_1 \rightarrow s_2$. $R$-confluence
requires some $s_3,t'$ such that $s_2 \StarOf{\rightarrow} s_3$ and $t
\StarOf{\rightarrow} t'$ and $R(s_3,t')$, but no such values exist.

To get around this issue, we require a different set of properties.
\begin{enumerate}
\item $(S,\rightarrow)$ must be $=$-confluent.
\item $R$ must be a bisimilulation relation for $(S,\StarOf{\rightarrow})$. In
  other words if $R(s_1,t_1)$, and $s_1 \StarOf{\rightarrow} s_2$, then there
  exists $t_2$ such that $t_1 \StarOf{\rightarrow} t_2$ and $R(s_2,t_2)$; and if
  $R(s_1,t_1)$, and $t_1 \StarOf{\rightarrow} t_2$, then there exists $s_2$ such
  that $s_1 \StarOf{\rightarrow} s_2$ and $R(s_2,t_2)$.
\end{enumerate}

\begin{theorem}
  \label{lem:right-side}
  Let $(S,\rightarrow)$ be $=$-confluent, and $R$ be a bisimilulation relation
  for $(S,\StarOf{\rightarrow})$. If $R(s_1,t_1)$ and $s_1 \StarOf{\rightarrow}
  s_2$ and $t_1 \StarOf{\rightarrow} t_2$, then there exists $s_3,t_3$ such that
  $s_2 \StarOf{\rightarrow} s_3$ and $t_2 \rightarrow t_3$ and $R(s_3,t_3)$.
\end{theorem}
\begin{proof}
  By induction length of the reduction of $s_1 \StarOf{\rightarrow} s_2$
  \begin{case}[Base Case]
    Let $R(s_1,t_1)$ and $t_1 \StarOf{\rightarrow} t_2$. As
    $R$ is a bisimulation relation for $(S,\StarOf{\rightarrow})$, there exists $s_2$ such that
    $s_1 \StarOf{\rightarrow} s_2$ and $R(s_2,t_2)$, as desired.
  \end{case}

  \begin{figure}
    \centering
    \includegraphics[scale=.4]{solved-proof.pdf}
    \caption{Diagram showing how we prove the inductive case of $R$-confluence
      of the transitive closure of a rewrite system. $R(s_4,t_4)$ comes from the
      inductive hypothesis. The existance of $s_5$ is guaranteed through
      $=$-confluence. $R(s_5,t_5)$ is guaranteed because $R$ is a bisimulation
      relation.}
    \label{fig:solved-proof}
  \end{figure}
  
  \begin{case}[Inductive Case]
    Let $R(s_1,t_1)$ and $s_1 \StarOf{\rightarrow} s_2$ and $s_2 \rightarrow
    s_3$ and $t_1 \StarOf{\rightarrow} t_2$. By the induction hypothesis, there
    exists $s_4,t_4$ such that $R(s_4,t_4)$ and $s_2 \StarOf{\rightarrow} s_4$
    and $t_2 \rightarrow t_4$.

    Because $(S,\rightarrow)$ is $=$-confluent, there
    exists $s_5$ such that $s_3 \StarOf{\rightarrow} s_5$ and $s_4
    \StarOf{\rightarrow} s_5$. As $(S,\StarOf{\rightarrow})$ is a bisimilulation
    relation on $(S,\StarOf{\rightarrow})$,
    there exists $t_5$ such that $t_4 \StarOf{\rightarrow} t_5$ and
    $R(s_5,t_5)$, as desired. This case is diagrammed in
    Figure~\ref{fig:solved-proof}.
  \end{case}
\end{proof}

\section{Related Work}

The concept of $R$-confluence is related to the notion of \emph{confluence
  modulo $\sim$}~\cite{huetconf}. The definition of confluent modulo $\sim{}$ is
almost the same as $\sim$-confluence, the only difference is that confluence
modulo $\sim{}$ requires $\sim{}$ to an equivalence relation. Conditions that suffice to prove a rewrite system confluent modulo
$\sim$ are not generally sufficient to prove $R$-confluence (and vice versa).
Furthermore, our bisimulation relations are closely related to local coherence
modulo $\sim{}$.

Bisimulation relations come from concurrency theory~\cite{bisimcoind}, but a
related notion, commuting rewrites~\cite{commut}, appear in the confluence
literature. We require a single rewrite of $R$ to commute with an arbitrary
number of rewrites of $\rightarrow$, which commuting rewrites do not express.

The full proof of completeness is contained in the appendix of the full version
of the original optician paper~\cite{extended-version}. The original proof of
$R$-confluence for the transitive closure of $\rightarrow$ required
additional assumptions. These unnecessary assumptions have been identified and
removed in this paper. Future work used the proof of completeness
over our lens language to show that quotient bijective lenses are also
synthesizeable~\cite{maina+:quotient-synthesis}. Lastly, while synthesizeability
was not proven for symmetric lenses~\cite{soptician}, such a proof would likely
have proven $R$-confluence in a similar manner.

This work continues a trend in making programming easier through
synthesis~\cite{gulwani2017program}. While synthesis is one approach to make
bidirectional programming easier, it is not the only approach. Work has
gone into building lenses without requiring a point-free combinator
style~\cite{10.1007/978-3-319-89884-1_2}. Other work has found
applicative~\cite{10.1145/2858949.2784750} and
monadic~\cite{10.1007/978-3-030-17184-1_6} approaches to compositionally
building lenses.


%------------------------------------------------------------------------------

\label{sect:bib}
\bibliographystyle{plain}
%\bibliographystyle{alpha}
%\bibliographystyle{unsrt}
%\bibliographystyle{abbrv}
\bibliography{local,bcp}

%------------------------------------------------------------------------------

%------------------------------------------------------------------------------
% Index
%\printindex

%------------------------------------------------------------------------------
\end{document}

