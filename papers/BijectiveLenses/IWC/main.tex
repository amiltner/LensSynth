% easychair.tex,v 3.5 2017/03/15

\documentclass{easychair}
%\documentclass[EPiC]{easychair}
%\documentclass[EPiCempty]{easychair}
%\documentclass[debug]{easychair}
%\documentclass[verbose]{easychair}
%\documentclass[notimes]{easychair}
%\documentclass[withtimes]{easychair}
%\documentclass[a4paper]{easychair}
%\documentclass[letterpaper]{easychair}

\usepackage{doc}

%\newif\ifdraft\drafttrue      % Uncomment to turn on comments
\newif\ifdraft\draftfalse   % Uncomment to turn off comments

% use this if you have a long article and want to create an index
% \usepackage{makeidx}

% In order to save space or manage large tables or figures in a
% landcape-like text, you can use the rotating and pdflscape
% packages. Uncomment the desired from the below.
%
% \usepackage{rotating}
% \usepackage{pdflscape}

% Some of our commands for this guide.
%
\newcommand{\easychair}{\textsf{easychair}}
\newcommand{\miktex}{MiK{\TeX}}
\newcommand{\texniccenter}{{\TeX}nicCenter}
\newcommand{\makefile}{\texttt{Makefile}}
\newcommand{\latexeditor}{LEd}

%\makeindex

%% Front Matter
%%
% Regular title as in the article class.
%
%
\title{Confluence in Lens Synthesis}

% Authors are joined by \and. Their affiliations are given by \inst, which indexes
% into the list defined using \institute
%
\author{
Anders Miltner\inst{1}
\and
Kathleen Fisher\inst{2}
\and
Benjamin C. Pierce\inst{3}
\and
David Walker\inst{4}
\and
Steve Zdancewic\inst{5}
}

% Institutes for affiliations are also joined by \and,
\institute{
  Princeton University\\
  \email{amiltner@cs.princeton.edu}
\and
   Tufts University\\
   \email{kfisher@eecs.tufts.edu}\\
\and
   University of Pennsylvania\\
   \email{bcpierce@cis.upenn.edu}\\
\and
   Princeton University\\
  \email{dpw@cs.princeton.edu}
\and
  University of Pennsylvania\\
  \email{stevez@cis.upenn.edu}\\
}

%  \authorrunning{} has to be set for the shorter version of the authors' names;
% otherwise a warning will be rendered in the running heads. When processed by
% EasyChair, this command is mandatory: a document without \authorrunning
% will be rejected by EasyChair

\authorrunning{Miltner, Fisher, Pierce, Walker and Zdancewic}

% \titlerunning{} has to be set to either the main title or its shorter
% version for the running heads. When processed by
% EasyChair, this command is mandatory: a document without \titlerunning
% will be rejected by EasyChair
\titlerunning{Confluence in Lens Synthesis}

\usepackage{amsmath}
\usepackage{mathtools}
\usepackage{bussproofs}
\usepackage{amsthm}
\usepackage{csvsimple}
\usepackage{thmtools,thm-restate}
\usepackage{changepage}
\usepackage{booktabs}
\usepackage{amssymb}
\usepackage[inline]{enumitem}
\usepackage{multirow,bigdelim}
\usepackage{multicol}
\usepackage{siunitx}
\usepackage{listings}
\usepackage{sansmath}
\usepackage{url}
\usepackage{flushend}
\usepackage{microtype}
\usepackage[utf8]{inputenc}
\usepackage{mathpartir}
\usepackage{empheq}
\usepackage{array}
\usepackage{pgfplots}
\usepackage{stmaryrd}
\usepackage{courier}
\usepackage{qtree}
\usepackage[normalem]{ulem}
\usepackage{relsize}
\usepackage{tikz}
\usepackage{algorithm}
\usepackage[noend]{algpseudocode}
\usepackage{graphicx}
\usepackage{subcaption}
\usepackage{textcomp}
\usepackage{tabularx}
\usepackage{stackengine}
\usepackage{caption}
\usepackage{wrapfig}
\usepackage{remreset}
\usepackage{tabulary}

\usetikzlibrary{
  er,
  matrix,
  shapes,
  arrows,
  positioning,
  fit,
  calc,
  pgfplots.groupplots,
  arrows.meta
}
\tikzset{>={Latex}}

\newcommand{\Regex}{\ensuremath{r}}         % Regular Expression
\newcommand{\UserDefined}{\ensuremath{U}}   % User Defined
\newcommand{\ExampledRegex}{\ensuremath{er}} % Exampled Regex
\newcommand{\Atom}{\ensuremath{a}}          % Atoms
\newcommand{\Clause}{\ensuremath{cl}}       % Clauses
\newcommand{\DNFRegex}{\ensuremath{dr}}         % Regular Expression
\newcommand{\Context}{\ensuremath{\Gamma}}  % Context
\newcommand{\TypeContext}{\ensuremath{\Delta}} %TypeContext
\newcommand{\String}{\ensuremath{s}}        % String
\newcommand{\ExampleNumberList}{\ensuremath{enl}} %Example Number List
\newcommand{\ExampleNumberListList}{\ensuremath{enll}}
\newcommand{\ExampleStringList}{\ensuremath{esl}}
\newcommand{\StringList}{\ensuremath{sl}}
\newcommand{\Natural}{\ensuremath{n}}
\newcommand{\Interleaving}[1]{\ensuremath{interleaving(#1)}}
\newcommand{\Interleave}{\ensuremath{interleave}}
\newcommand{\BinaryInterleave}[2]{\ensuremath{interleave(#1,#2)}}
\newcommand{\NAryInterleave}[2]{\ensuremath{interleave(#1,\ldots,#2)}}
\newcommand{\Combine}{\ensuremath{combine}}
\newcommand{\List}{\ensuremath{l}}
\newcommand{\ValidCombine}[2]{\ensuremath{validcombine(#1,#2)}}
\newcommand{\Parent}[1]{\ensuremath{parent(#1)}}
\newcommand{\Parented}[1]{\ensuremath{parented(#1)}}
\newcommand{\CombineString}[1]{\ensuremath{combine_{\ExampleStringList}(#1)}}
\newcommand{\CombineList}[1]{\ensuremath{combine_{\ExampleNumberListList}(#1)}}
\newcommand{\Length}[1]{\ensuremath{len(#1)}}


\newcommand{\Lens}{\ensuremath{l}}
\newcommand{\AtomLens}{\ensuremath{al}}
\newcommand{\ClauseLens}{\ensuremath{cll}}
\newcommand{\DNFLens}{\ensuremath{dl}}


\newcommand{\Star}[1]{\ensuremath{#1^*}}
\newcommand{\ConcatLens}[2]{\ensuremath{concat(#1,#2)}}
\newcommand{\SwapLens}[2]{\ensuremath{swap(#1,#2)}}
\newcommand{\OrLens}[2]{\ensuremath{or(#1,#2)}}
\newcommand{\IdentityLens}{\ensuremath{identity}}
\newcommand{\IterateLens}[1]{\ensuremath{iterate(#1)}}
\newcommand{\ComposeLens}[2]{\ensuremath{#1 \circ #2}}

% GRAMMAR OPERATORS
\newcommand{\GBar}{\ensuremath{~|~}}
\newcommand{\GIndent}{\hspace{.5in}}
\newcommand{\GEq}{\ensuremath{::=~}}  
\newcommand{\GEmp}{\ensuremath{\cdot}}



\begin{document}

\maketitle

\begin{abstract}
  Bidirectional programs are programs where a single expression has multiple
  interpretations, encoding a number of functions with a single piece of syntax.
  Lenses are bidirectional programs with semantic guarantees, and lens languages
  have been developed for synchronizing string formats, building updatable views
  in databases, and more. These languages often are typed, where a well-typed
  expression will provide the semantic guarantees. Recent research has shown how
  string lenses can be synthesized from their types, pairs of regular
  expressions. However, providing completeness guarantees is quite tricky on
  these languages, due in large part to the many equivalences on regular
  expressions.

  In proving a completeness guarantee, a confluence-like property,
  $R$-confluence, must be shown. This property can be thought of as a
  generalization of standard confluence. Where standard confluence is
  parameterized by a set $S$ and a binary relation $\rightarrow$, indexed
  confluence is parameterized by an additional binary relation $R \subseteq S
  \times S$. In this model, confluence is the specific case where $R$ is just
  equality. We show how the proof that the transitive closure of a rewrite
  system is $R$-confluent differs from the proof strategy for standard
  confluence.
\end{abstract}

\section{Introduction}

Bidirectional transformations are pervasive in modern software systems, occuring
as database views and view updaters, parsers and pretty-printers, data
synchronization tools, and more. Instead of manually building the functions that
comprise a bidirectional transformation, programmers can build them both ``at
once'' using a bidirectional programming language. Bidirectional programming
languages have been developed for creating view
updaters~\cite{BohannonPierceVaughan}, Linux configuration file
editors~\cite{augeas2}, direct manipulation programming
systems~\cite{bidirectionaleval}, and
more~\cite{DBLP:conf/pepm/KoZH16,DBLP:conf/icfp/HidakaHIKMN10,DBLP:conf/staf/ZhuK0SH15}. Lenses are a
particularly well-behaved class of bidirectional programs, where the underlying
transformations are guaranteed to satisfy a number of ``round-tripping'' laws.
Lens-based bidirectional programming languages often provide round-tripping
guarantees through a set of typing rules; well-typed lens expressions are
guaranteed to satisfy the round-tripping laws.

To make bidirectional programming easier, Optician~\cite{optician} synthesizes
bidirectional programs from these types, specifically in Boomerang, a language
for synchronizing string data. Optician takes as input two regular expressions
(which serve as the type of a Boomerang lens) and a set of examples specifying
input-output behavior, and synthesizes a well-typed lens between the languages
of those regular expressions.

The judgement $\Lens : \Regex \Leftrightarrow \RegexAlt$ means that $\Lens$
bidirectionally converts between $\LanguageOf{\Regex}$ and
$\LanguageOf{\RegexAlt}$. For synthesis, given $\Regex$ and $\RegexAlt$, we must
find a lens $\Lens$ such that $\Lens : \Regex \Leftrightarrow \RegexAlt$. There
are a number of typing rules for lenses, but the rules for type-equivalence and
composition (shown below) do not permit an efficient search algorithm. Note that
in type equivalence we use a finer notion of equivalence than language
equivalence, we instead use star-semiring equivalence, denoted $\SSREquiv$. We were unable to make
our proof strategy work for language equivalence, so instead guarantee
completeness only on the subset of lenses well-typed with star-semiring
equivalence.
\begin{mathpar}
  \inferrule[\RewriteRegexLensRule{}]
  {
    \Lens \OfType \Regex \Leftrightarrow \RegexAlt\\
    \Regex \SSREquiv \Regex'\\
    \RegexAlt \SSREquiv \RegexAlt'
  }
  {
    \Lens \OfType \Regex' \Leftrightarrow \RegexAlt'
  }
  
  \inferrule[\ComposeLensRule{}]
  {
    \Lens_1 \OfType \Regex_1 \Leftrightarrow \Regex_2\\
    \Lens_2 \OfType \Regex_2 \Leftrightarrow \Regex_3\\
  }
  {
    \ComposeLensOf{\Lens_1}{\Lens_2} \OfType \Regex_1 \Leftrightarrow \Regex_3
  }
\end{mathpar}

Trying to synthesize a lens from the types by applying typing rules ``in
reverse'' gets confounded by these two rules. They require no significant
restriction on the output types, and so can be applied at any time. Furthermore,
applying them in reverse requires pulling regular expressions ``out of thin
air.'' For \RewriteRegexLensRule, these regular expressions need only be
equivalent to the input regular expressions, and for \ComposeLensRule, there are
no restrictions at all!

Our solution is to synthesize in an alternative language of disjunctive normal
form lenses (DNF lenses), typed by disjunctive normal form regular expressions
(DNF REs). Both DNF lenses and DNF REs can be thought of as lenses and regular
expressions, put into a pseudonormal form.

\begin{figure}
  \centering
  \begin{tikzpicture}[auto,node distance=1.5cm]
    \node[text width=1.5cm,minimum height=.6cm,align=center,draw,rectangle] (todnfregex) {\ToDNFRegex{}};
    
    \node[align=right, anchor=east] (regex1) [left = .6cm of todnfregex.north west]{\Regex{}};
    \node[align=right, anchor=east] (regex2) [left = .6cm of todnfregex.south west]{\RegexAlt{}};
    \node[align=right, anchor=east] (exs) [below = .2cm of regex2]{ \Examples{} };
    
    \node[align=center] (dnfregex1) [right = .4cm of todnfregex.north east]{\DNFRegex{}};
    \node[align=center] (dnfregex2) [right = .4cm of todnfregex.south east]{\DNFRegexAlt{}};

    \node[text width=2.9cm,minimum height=.6cm,align=center,draw,rectangle] [right = 1.45cm of todnfregex.east] (synthdnflens) {\SynthDNFLens{}};
    \node[align=center] [above = .7cm of synthdnflens] (optician) {\Optician{}};
    
    \node[align=center] [right = .4cm of synthdnflens] (dnflens) {\DNFLens{}};
    
    \node[text width=1.5cm,minimum height=.6cm,align=center,draw,rectangle] [right = .4cm of dnflens] (tolens) {\ToLens{}};
    
    \node[align=center] [right = .6cm of tolens] (lens) {\Lens{}};
    
    
    \path[->] (regex1.east) edge (todnfregex.north west);
    \path[->] (regex2.east) edge (todnfregex.south west);
    
    \path[->] (todnfregex.north east) edge (dnfregex1.west);
    \path[->] (todnfregex.south east) edge (dnfregex2.west);
    
    \path[->] (dnfregex1.east) edge (synthdnflens.north west);
    \path[->] (dnfregex2.east) edge (synthdnflens.south west);
    
    \path[->] (synthdnflens) edge (dnflens);
    
    \path[->] (dnflens) edge (tolens);
    
    \path[->] (tolens) edge (lens);
    \draw[->] ($(exs.east)+(-3pt,0)$) -| node(exsedge) {} (synthdnflens);
    \node[fit={($(todnfregex.west)+(-4pt,0)$) ($(tolens.east)+(4pt,0)$) (exsedge) (dnfregex1) (optician) (dnfregex2)},draw] (surrounding) {};
    % Now place a relation (ID=rel1)
    %\node[text width=2cm,align=center,draw, rectangle] (sketch-gen) [right = .75cm of spec] {\TypeProp{}};
    %\node (below-gen) [below=.5cm of sketch-gen] {};
    %\node[text width=2cm,align=center,draw, rectangle] (sketch-compl)
    %     [right = .25cm of sketch-gen] {\RigidSynth{}};
    %\node (below-compl) [below=.5cm of sketch-compl] {};
    %\node[align=center] (lens) [right = .75cm of sketch-compl] {Lens}; 
    %% Draw an edge between rel1 and node1; rel1 and node2
    %\path[->] (spec) edge node (start-alg) {} (sketch-gen);
    %\path[->] (sketch-gen) edge node(middle) {} (sketch-compl);
    %\path[->] (sketch-compl) edge node[near start](success) {\Success{}} (lens);
    %\draw[<-] (sketch-gen.south) -- +(0,-.5) -| node[above left](failure){\Failure{}} (sketch-compl.south);

    %\node (synth-name) [above=.5cm of middle] {\Optician{}};
    %
    %\node[fit=(sketch-gen) (sketch-compl) (start-alg) (synth-name) (failure) (success) ,draw] (surrounding) {};
  \end{tikzpicture}
  \caption{Schematic Diagram for \Optician{}.  Regular expressions, \Regex{} and
    \RegexAlt{}, and examples, \Examples{}, are given as input.
    First, the function \ToDNFRegex{} converts \Regex{} and \RegexAlt{} into
    their respective DNF forms, \DNFRegex{} and \DNFRegexAlt{}.
    Next, \SynthDNFLens{} synthesizes a DNF lens, \DNFLens{}, from \Regex{},
    \RegexAlt{}, and \Examples{}.
    Finally, \ToLens{} converts \DNFLens{} into \Lens{}, a lens in Boomerang
    that is equivalent to \DNFLens{}.}
  \label{fig:schematic-diagram-synthesis}
\end{figure}

Two typing judgements are involved in typing a DNF lens, $\DNFLens : \DNFRegex
\Leftrightarrow \DNFRegexAlt$ and $\DNFLens \OfRewritelessType \DNFRegex
\Leftrightarrow \DNFRegexAlt$. In both instances, the judgement means $\DNFLens$
is a DNF lens that converts between $\LanguageOf{\DNFRegex}$ and
$\LanguageOf{\DNFRegexAlt}$. If $\DNFLens \OfRewritelessType \DNFRegex
\Leftrightarrow \DNFRegexAlt$, then the typing derivation of $\DNFLens$ never
rewrites the DNF REs. The only typing rule for $\DNFLens : \DNFRegex
\Leftrightarrow \DNFRegexAlt$ is:
\begin{mathpar}
  \inferrule[\RewriteDNFRegexLensRule{}]
  {
    \DNFRegex' \StarOf{\Rewrite} \DNFRegex\\
    \DNFRegexAlt' \StarOf{\Rewrite} \DNFRegexAlt \\
    \DNFLens \OfRewritelessType \MapsBetweenTypeOf{\DNFRegex}{\DNFRegexAlt}
  }
  {
    \DNFLens \OfType \MapsBetweenTypeOf{\DNFRegex'}{\DNFRegexAlt'}
  }
\end{mathpar}
To type a DNF lens, one first applies a number of rewrites to the DNF REs, then
builds up a DNF lens using rewriteless typing rules. Because of this
phase distinction, equivalent
types only need to be generated once, and composition never need be applied.
This structure corresponds to an efficient algorithm that first searches through type
equivalences with rewrites and then builds up a rewriteless DNF
lens as directed by the syntax of the DNF RE. 
Figure~\ref{fig:schematic-diagram-synthesis} shows the full synthesis
algorithm. The input
regular expressions are converted into DNF form using the
$\ToDNFRegex$ function. Next, a
well-typed DNF lens is synthesized and then converted back into a standard lens
language with the $\ToLens$ function.

Proving this search procedure complete reduces to proving DNF lenses complete
with respect to our standard lens language. In particular, we wish the prove the
following theorem:
\begin{theorem}
  If $\Lens : \Regex \Leftrightarrow \RegexAlt$, then there exists a DNF lens,
  $\DNFLens$, such that $\DNFLens \OfType \ToDNFRegexOf{\Regex} \Leftrightarrow
  \ToDNFRegexOf{\RegexAlt}$ and the semantics of $\Lens$ and $\DNFLens$
  are equivalent.
\end{theorem}

We prove this propety by induction. The particular difficulty lies in the lens
equivalence rule. We begin this case below:
  
\begin{mathpar}
  \inferrule
  {
    \Lens \OfType \Regex_1 \Leftrightarrow \RegexAlt_1\\
    \Regex \SSREquiv \Regex_2\\
    \RegexAlt \SSREquiv \RegexAlt_2
  }
  {
    \Lens \OfType \Regex_2 \Leftrightarrow \RegexAlt_2
  }
\end{mathpar}

By induction assumption, there exists $\DNFLens \OfType \ToDNFRegexOf{\Regex_1}
\Leftrightarrow \ToDNFRegexOf{\RegexAlt_1}$, where the semantics of $\DNFLens$
are equivalent to those of $\Lens$. By inversion
\begin{mathpar}
  \inferrule
  {
    \ToDNFRegexOf{\Regex_1} \StarOf{\Rewrite} \DNFRegex_1\\
    \ToDNFRegexOf{\RegexAlt_1} \StarOf{\Rewrite} \DNFRegexAlt_1\\
    \DNFLens \OfRewritelessType \DNFRegex_1 \Leftrightarrow \DNFRegexAlt_1
  }
  {
    \DNFLens \OfType \ToDNFRegexOf{\Regex_1} \Leftrightarrow \ToDNFRegexOf{\RegexAlt_1}
  }
\end{mathpar}

To complete this case, we need to find a DNF lens $\DNFLens' \OfType
\ToDNFRegexOf{\Regex_2} \Leftrightarrow \ToDNFRegexOf{\RegexAlt_2}$ with
equivalent semantics to $\Lens$.

We first show that there exists $\DNFRegex_2$ and $\DNFRegexAlt_2$ such that
$\ToDNFRegexOf{\Regex_2} \StarOf{\rightarrow} \DNFRegex_2$ and
$\ToDNFRegexOf{\DNFRegex_1} \StarOf{\rightarrow} \DNFRegex_2$ and
$\ToDNFRegexOf{\RegexAlt_2} \StarOf{\rightarrow} \DNFRegexAlt_2$ and
$\ToDNFRegexOf{\DNFRegexAlt_1} \StarOf{\rightarrow} \DNFRegexAlt_2$. To do this,
we first prove that $\StarOf{\rightarrow}$ is confluent, and that if $\Regex
\SSREquiv \RegexAlt$, then $\ToDNFRegexOf{\Regex}$ is related to
$\ToDNFRegexOf{\RegexAlt}$ through the transitive, reflexive closure of
$\rightarrow$. We then apply a corollary of confluence to show the existance of
such a $\DNFRegex_2$ and $\DNFRegexAlt_2$

\begin{figure}
  \centering
    \includegraphics[scale=.4]{equiv-case.pdf}
    \caption{Diagram constructing a well-typed DNF lens between
      \ToDNFRegexOf{\Regex_2} and \ToDNFRegexOf{\RegexAlt_2}.}
    \label{fig:equiv-case}
\end{figure}

After this, we have $\DNFLens \OfRewritelessType \DNFRegex_1 \Leftrightarrow
\DNFRegexAlt_1$ and $\DNFRegex_1 \StarOf{\rightarrow} \DNFRegex_2$ and
$\DNFRegexAlt_1 \StarOf{\rightarrow} \DNFRegexAlt_2$. If we can prove a
confluence-like property that would show the existance of some $\DNFLens'
\OfRewritelessType \DNFRegex_3 \Leftrightarrow \DNFRegexAlt_3$, where
$\DNFRegex_2 \StarOf{\rightarrow} \DNFRegex_3$ and $\DNFRegexAlt_2
\StarOf{\rightarrow} \DNFRegexAlt_3$, we would be done. This property is
$R$-confluence for a properly chosen $R$. This case is diagrammed in Figure~\ref{fig:equiv-case}

\section{$R$-Confluence Formulation}
Let $S$ be an underlying set, and $\rightarrow$ and $R$ be binary relations. The
indexed rewrite system $(S,\rightarrow)$ is $R$-confluent, if for all $s_1,s_2 \in
S$, if $R(s_1,s_2)$, $s_1 \rightarrow s_1'$, and $s_2 \rightarrow s_2'$, then
there exist $s_1''$ and $s_2''$ such that $R(s_1'',s_2'')$, $s_1' \rightarrow
s_1''$, and $s_2 \rightarrow s_2''$.

In the context of lens synthesis, $S$ is the set of DNF REs,
the rewrites are $\StarOf{\rightarrow}$, and given a DNF lens $\DNFLens$,
$R_\DNFLens(\DNFRegex,\DNFRegexAlt)$ is true if there exists a DNF lens $\DNFLens'$ such
that $\DNFLens' \OfRewritelessType \DNFRegex \Leftrightarrow \DNFRegexAlt$ and
$\DNFLens$ is equivalent to $\DNFLens'$.

If we followed the standard way to prove a rewrite system confluent, we would
find a set of rewrite rules $\ParallelRewrite$ such that
$\StarOf{\ParallelRewrite} = \StarOf{\Rewrite}$ and $(S,\ParallelRewrite)$ is
$R_\DNFLens$-confluent. Then, we need to prove that
$(S,\StarOf{\ParallelRewrite})$ is $R_\DNFLens$-confluent from knowledge that
$(S,\ParallelRewrite)$ is $R_\DNFLens$-confluent. However, we will show this
approach does not work.

\section{Proving (S,\StarOf{\rightarrow}) $R$-Confluent}

\begin{figure}
  \centering
    \includegraphics[scale=.4]{proof-attempt.pdf}
    \caption{If trying to prove $R$-confluence for the transitive closure of a
      rewrite system in the standard way, we get stuck at this point in the
      inductive case.}
    \label{fig:proof-attempt}
\end{figure}

The standard way of proving $(S,\StarOf{\rightarrow})$ to be $=$-confluent is to
first prove $(S,\rightarrow)$ to be $=$-confluent, then to perform two
inductions, first on the left arrows then on the right arrows. This approach
does not work for arbitrary $R$-confluence. Consider the inductive case, where
$R(s_1,t_1)$ and $s_1 \StarOf{\rightarrow} s_2$ and $s_2 \rightarrow s_3$ and
$t_1 \StarOf{\rightarrow} t_2$, shown in Figure~\ref{fig:proof-attempt}. By
applying the induction hypothesis, we know there exists $s_4, t_4$ such that $s_2
\rightarrow s_4$ and $t_2 \StarOf{\rightarrow} t_4$, and $R(s_4,t_4)$. We know
that $s_2 \rightarrow s_3$ and $s_2 \rightarrow s_4$, but we only know that
$(S,\rightarrow)$ is $R$-confluent, and so we are now stuck.

To get around this issue, we require a different set of properties.
\begin{enumerate}
\item $(S,\rightarrow)$ must be $=$-confluent.
\item $(S,\rightarrow)$ must be \emph{$R$-bisimilar}. In other words,
  if $R(s_1,t_1)$, and $s_1 \rightarrow s_2$, then there exists $t_2$ such that
  $t_1 \rightarrow t_2$ and $R(s_2,t_2)$; and if $R(s_1,t_1)$, and $t_1
  \rightarrow t_2$, then there exists $s_2$ such that $s_1 \rightarrow s_2$ and
  $R(s_2,t_2)$.
\end{enumerate}

Now we can proceed with our proof.

\begin{theorem}
  \label{lem:right-side}
  Let $(S,\rightarrow)$ be $=$-confluent, and $(S,\rightarrow)$ be
  $R$-bisimilar. If $R(s_1,t_1)$ and $s_1 \StarOf{\rightarrow} s_2$ and $t_1
  \StarOf{\rightarrow} t_2$, then there exists $s_3,t_3$ such that $s_2
  \StarOf{\rightarrow} s_3$ and $t_2 \rightarrow t_3$ and $R(s_3,t_3)$.
\end{theorem}
\begin{proof}
  By induction on $s_1 \StarOf{\rightarrow} s_2$
  \begin{case}[Base Case]
    Let $R(s_1,t_1)$ and $t_1 \StarOf{\rightarrow} t_2$. As
    $(S,\StarOf{\rightarrow})$ is $R$-bisimilar, there exists $s_2$ such that
    $s_1 \StarOf{\rightarrow} s_2$ and $R(s_2,t_2)$, as desired.
  \end{case}

  \begin{figure}
    \centering
    \includegraphics[scale=.4]{solved-proof.pdf}
    \caption{Diagram showing how we prove the inductive case of $R$-confluence
      of the transitive closure of a rewrite system. $R(s_4,t_4)$ comes from the
      inductive hypothesis. The existance of $s_5$ is guaranteed through
      $=$-confluence. $R(s_5,t_5)$ comes from $R$-bisimilarity.}
    \label{fig:solved-proof}
  \end{figure}
  
  \begin{case}[Inductive Case]
    Let $R(s_1,t_1)$ and $s_1 \StarOf{\rightarrow} s_2$ and $s_2 \rightarrow
    s_3$ and $t_1 \StarOf{\rightarrow} t_2$. By the induction hypothesis, there
    exists $s_4,t_4$ such that $R(s_4,t_4)$ and $s_3 \StarOf{\rightarrow} s_4$
    and $t_3 \rightarrow t_4$.

    As $(S,\rightarrow)$ is $=$-confluent, we know from prior work
    $(S,\StarOf{\rightarrow})$ is $=$-confluent. Because of this, there exists
    $s_5$ such that $s_3 \StarOf{\rightarrow} s_5$ and $s_4 \StarOf{\rightarrow}
    s_5$. As $(S,\StarOf{\rightarrow})$ is $R$-bisimilar, there exists $t_5$
    such that $t_4 \StarOf{\rightarrow} t_5$ and $R(s_5,t_5)$, as desired. This
    case is diagrammed in Figure~\ref{fig:solved-proof}.
  \end{case}
\end{proof}

%------------------------------------------------------------------------------

\label{sect:bib}
\bibliographystyle{plain}
%\bibliographystyle{alpha}
%\bibliographystyle{unsrt}
%\bibliographystyle{abbrv}
\bibliography{local,bcp}

%------------------------------------------------------------------------------

%------------------------------------------------------------------------------
% Index
%\printindex

%------------------------------------------------------------------------------
\end{document}

